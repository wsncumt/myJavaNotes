\documentclass[a4paper]{report}
\usepackage[space,fancyhdr,fntef]{ctexcap}
\usepackage{fontspec}
\fontspec{宋体}
\setmainfont{Times New Roman}
%\fontsize{50pt}{50pt}\selectfont
\renewcommand{\rmdefault}{ptm}
\usepackage[namelimits,sumlimits,nointlimits]{amsmath}
\usepackage[text={169mm,250mm},bottom=20mm,top=25mm,left=25mm,right=15mm,centering]{geometry}
\usepackage{color}
\usepackage{CJKfntef}%下划线宏包160页
\usepackage{xcolor}
\usepackage{arydshln}%234页,虚线表格宏包
\pagestyle{fancy} \fancyhf{}
\fancyhead[OC]{\color{gray}\rightmark}

\fancyhead[EC]{\color{gray}\leftmark}
\fancyfoot[C]{\color{gray}\thepage}
\renewcommand{\headrule}{\color{gray}\hrule width\headwidth}
%\renewcommand{\footrulewidth}{0.4pt}%改为0pt即可去掉页脚上面的横线
%\usepackage{parskip}
%\usepackage{indentfirst}
\usepackage{graphicx}%插图宏包,参见手册318页
\definecolor{dkgreen}{RGB}{106,135,89}
\definecolor{dkblue}{RGB}{103,150,163}
\definecolor{wgray}{RGB}{248,248,248}
\definecolor{WGRAY}{RGB}{248,248,248}
\usepackage{listings}
\lstset{language=Java,
backgroundcolor=\color{wgray},
rulesepcolor=\color{red!20!green!20!blue!20},%代码块边框为淡青色
%lablestep=1,
%lablesep=5pt,
%lablestyle=\tiny,
%tablesize=4,
%captionpos=b,
basicstyle=\ttfamily\small,
keywordstyle=\color{orange},
commentstyle=\color{gray},
stringstyle=\color{dkgreen},
numberstyle=\tiny,
numbersep=8pt,
frame=single,%topline.bottomline,lines,single,leftline
identifierstyle=\color{dkblue},
numbers=left,
stepnumber=1,
xleftmargin=2em,xrightmargin=2em, aboveskip=1em,
breaklines=true
}
\usepackage[xetex,colorlinks]{hyperref}%394页  \href{网址}{文本}
\hypersetup{urlcolor=blue}
%\linebreak[2]%换行,152页
\usepackage{fancybox}%盒子宏包55页
\setcounter{secnumdepth}{4}
\CTEXoptions[contentsname={目\hspace{15pt}录}]
\CTEXsetup[beforeskip={-40pt},afterskip={20pt plus 2pt minus 2pt}]{chapter}

%目录设置
\usepackage{titletoc}
\usepackage[toc]{multitoc}
\titlecontents{chapter}[4em]{\addvspace{2.3mm}\bf}{\contentslabel{4.0em}}{}{\titlerule*[5pt]{$\cdot$}\contentspage}
\titlecontents{section}[4em]{}{\contentslabel{2.5em}}{}{\titlerule*[5pt]{$\cdot$}\contentspage}
\titlecontents{subsection}[7.2em]{}{\contentslabel{3.3em}}{}{\titlerule*[5pt]{$\cdot$}\contentspage}
\usepackage{fancyvrb}%75页抄录宏包
\begin{document}
\flushbottom%版心底部对齐
\newcommand{\dm}[1]{\colorbox{wgray}{\lstinline`#1`}}
\newcommand{\myroman}[1]{\uppercase\expandafter{\romannumeral#1}}
\newcounter{num}[section] \renewcommand{\thenum}{\arabic{num}.} \newcommand{\num}{\refstepcounter{num}\text{\thenum}}

\newenvironment{tips}{\kaishu\zihao{-6}\color{blue}{\noindent\rule[-3pt]{\textwidth}{0.5pt}\par \em \noindent {\zihao{-5} \textcolor[rgb]{1.00,0.00,0.00}{Tips}}}\par}{\\ \rule[3mm]{\textwidth}{0.5pt}\par}

\newenvironment{zhengming}{\kaishu\zihao{-5}\color{blue}{\noindent\em 证明:}\par}{\hfill $\diamondsuit$\par}

\tableofcontents
\pagenumbering{Roman}%设置目录页码
\clearpage
\pagenumbering{arabic}%设置正文页码

\chapter{I/O编程}
\section{字节流}
\dm{InputStream}和\dm{OutputStream}是所有字节输入输出流的父类,都为抽象类。
\section{\dm{FileInputStream}和\dm{FileOutputStream}}
文件字节流,是节点流。
\subsection{\dm{FileInputStream}}
写入一个文件。
\begin{lstlisting}
InputStream in = new FileInputStream("E://a.txt");//文件不存在,抛出异常
\end{lstlisting}
\begin{tips}
\dm{InputStream}

1.\dm{int read()}:读一个字节,返回值为读到的字节数据。每读一次,指针都会向后移动一个字节。如果该方法返回-1表示读取结束。

2.\dm{int read(bytes[] bs)}:从文件读取字节,把数组\dm{bs}读满,返回值为读到的字节个数。返回-1表示读取结束。

3.\dm{int read(bytes[] bs, int start, int length)}:从文件读取字节,尝试读满数组中的一段(把数组的从\dm{start}开始长\dm{length}的区域读满),返回值为读到的字节个数。返回-1表示读取结束。
\end{tips}
\subsection{\dm{FileOutputStream}}
\dm{FileInputStream}:读取一个文件。
\begin{lstlisting}
String filename = "d:\\a.txt";
OutputStream out = new FileOutputStream(filename);//要记得把文件名传入构造器,不写true表示覆盖原来的文件
OutputStream out = new FileOutputStream(filename,true);//要记得把文件名传入构造器,写true表示在原来的文件进行追加
out.write('A');//写入一个字节
out.close();
\end{lstlisting}

\begin{tips}
\dm{OutputStream}

字符串转字节数组:\dm{str.getBytes();}

1.\dm{write(int a)}:往文件写出一个字节

2.\dm{write(byte[] bs)}:将字节数组\dm{bs}所有的数据写入文件

3.\dm{write(byte[] bs, int start, int length)}:将字节数组\dm{bs}从\dm{start}开始往文件写入\dm{length}个字节。
\end{tips}

\begin{lstlisting}[title=异常处理]
String filename = "d:\\a.txt";
OutputStream out = null;
//OutputStream out = new FileOutputStream(filename,true);//要记得把文件名传入构造器,写true表示在原来的文件进行追加
try{
    out = new FileOutputStream(filename);//要记得把文件名传入构造器,不写true表示覆盖原来的文件
    out.write('A');//写入一个字节
}catch(Exception e){
    System.out.println(e.getMessage());
}finally{
    try{
        out.close();
    }catch(Exception e1){
        System.out.println(e1.getMessage());
    }
}

\end{lstlisting}
\subsection{案例:文件拷贝}
从\dm{src}拷贝到\dm{dst}
\begin{lstlisting}[title=版本1]
//文件拷贝
//long a = System.nanoTime();//纳秒
//long b = System.nanoTime();
//System.out.println((b-a)/1e9);
public static void fileCopy1(String src,String dst){
    InputStream fis = null;
    OutputStream fos = null;
    try {
        fis = new FileInputStream(src);
        fos = new FileOutputStream(dst);
        byte[] bs = new byte[1024*4];//一页4个kb
        while(true){
            int len = fis.read(bs);
            if(len == -1){
                break;
            }
            fos.write(bs,0,len);
        }
    }catch(Exception e1){
        System.out.println(e1.getMessage());
    }finally {
        try{
            fis.close();
        }catch(Exception e2){
            System.out.println(e2.getMessage());
        }
        try{
            fos.close();
        }catch(Exception e2){
            System.out.println(e2.getMessage());
        }
    }
}
\end{lstlisting}
\subsection{过滤流}
对于int、long这类的无法直接转为字节数组,需要使用过滤流来完成读写。

\dm{DataOutputStream},过滤流必须由节点流来构建,即:
\begin{lstlisting}
OutputStream fos = null;
DataOutputStream dfos;
try {
    fos = new FileOutputStream("a.txt");
    dfos = new DataOutputStream(fos);
} catch (Exception e) {
    System.out.println(e.getMessage());
}finally{
    try{
        dfos.close();
        fos.close();
    }catch(Exception e1){
        System.out.println(e1.getMessage());
    }
}
\end{lstlisting}
\begin{tips}
\dm{DataOutputStream}:往文件写入8种基本类型和\dm{String}

例如:写入\dm{long},\dm{dfos.writeLong(10l)}
\end{tips}
\subsection{I/O操作流程}
1.创建节点流;

2.包装过滤流;

3.读写数据

4.关闭流
\subsection{缓冲流}
缓冲流(\dm{BufferedOutputStream/BufferedInputStream})为过滤流。

作用:利用一个缓冲区,提高I/O效率,减少访问磁盘的次数。

\dm{flush()}方法是将缓存区文件内容写入文件中。写方法与文件输出流一样。
\begin{lstlisting}
OutputStream fos = null;
BufferedOutputStream bos;
try {
    fos = new FileOutputStream("a.txt");
    bos = new BufferedOutputStream(fos);
    bos.writeLong(10l);//数据写入缓冲流
    bos.flush();//清空缓冲区,将缓存区文件内容写入文件中。
}
//catch和关闭流略
\end{lstlisting}
\subsubsection{文件拷贝}
\begin{lstlisting}[title=使用缓冲流拷贝文件]
//文件拷贝2
//加字节数组减少了I/O操作
//加缓冲流减少了访问磁盘操作
public static void fileCopy2(String src, String dst) {
    InputStream fis = null;
    BufferedInputStream bis = null;
    OutputStream fos = null;
    BufferedOutputStream bos = null;
    try {
        fis = new FileInputStream(src);
        bis = new BufferedInputStream(fis);
        fos = new FileOutputStream(dst);
        bos = new BufferedOutputStream(fos);
        byte[] bs = new byte[1024 * 4];
        while (true) {
            int len = bis.read(bs);
            if (len == -1) {
                break;
            }
            bos.write(bs, 0, len);
        }
    } catch (Exception e1) {
        System.out.println(e1.getMessage());
    } finally {
        try {
            bos.close();
            fos.close();
            bis.close();
            fis.close();
        } catch (Exception e2) {
            System.out.println(e2.getMessage());
        }
    }
}
\end{lstlisting}

\subsection{\dm{ObjectOutPutStream/ObjectInputStream}}
过滤流,增强了缓冲区功能,增强了读写8种数据类型和字符串功能。

增强了读写对象的功能:\dm{readObject()}从流中读取一个对象,\dm{writeObject(Object obj)}向流中写入一个对象。

\begin{tips}
通过流传输对象:对象的序列化。

只有实现了\dm{Serializable}接口的对象才能序列化。\dm{Serializable}接口中没有方法。实现该接口的类不需要添加额外的方法。

\dm{transient}关键字为属性的修饰符,用该关键字修饰的属性属于临时属性,不参与对象的序列化。(往文件写时不会存该属性,从文件读取时该属性置为其对应的默认值。)
\end{tips}

\subsubsection{读取整个文件}
\dm{ObjectInputStream}的读取方法\dm{readObject()}返回值不是数,因此无法通过-1来判断是否读完文件,文件读取完毕后,该方法会抛出一个\dm{EFOException}异常,可以通过捕获该异常来结束读取文件。
\begin{lstlisting}
FileInputStream fis = null;
ObjectInputStream ois = null;
try{
    fis = new FileInputStream("a.txt");
    ois = new ObjectInputStream(ois);
    try{
        while(true){
            System.out.println(ois.readObject());
        }
    }catch(EFOException e){
        System.out.println("文件读取结束!");
    }
}catch(Exception e1){
    System.out.println(e1.getMessage());
}finally{
    ois.close();
    fis.close();
}
\end{lstlisting}

\dm{fis.available();}返回源文件包含的字节数。
\subsubsection{反序列化}
如果该对象的类实现了\dm{Serializable}接口,反序列化时无需通过该类的构造方法。

如果该对象的类没有实现\dm{Serializable}接口,反序列化时需通过该类的构造方法(必须是无参的构造方法,否则会抛出异常)。
\subsection{自定义序列化规则}
实现\dm{Externalizable}接口,对该接口的两个方法进行重写。
%\begin{lstlisting}
%
%\end{lstlisting}

\begin{Verbatim}[frame=single,numbersep=5pt,xleftmargin=1.5em,xrightmargin=1.5em]
class A implements Externalizable {
    int age;
    String name;

    public A() {
    }
    //Externalizable方法
    @Override
    public void writeExternal(ObjectOutput out) throws IOException {
        out.writeUTF(name);
        out.writeInt(age);
    }

    @Override
    public void readExternal(ObjectInput in) throws Exception {
        name = in.readUTF();
        age = in.readInt();
    }
}
\end{Verbatim}

注:\dm{readExternal}方法抛出的异常为\dm{throws IOException ClassNotFoundException}
\subsection{重复序列化}
如果对一个对象重复多次序列化后存入文件,则文件只会存入该对象第一次序列化时所有的信息,后边的都存的该对象的相关引用。

导致的问题,如果先存入该对象,但该对象某些属性改变后,想继续往文件写入该对象,存入的依旧是相关引用。

解决方法:创建一个与当前对象参数完全一样的对象,只改变需要改变的属性。

\dm{clone()}方法是\dm{Object}类的,因此所有的类都包括该方法,但该方法是\dm{protected},使用类的\dm{clone()}方法需要在该类中对其进行重写。

对象克隆其对应的类还需实现\dm{Cloneable}接口,该接口没有方法。

\begin{lstlisting}
class A implements Cloneable{
    int age;
    String name;

    public A() {
    }

    //其他一些属性和方法
    //仅仅将权限改为public
    @Override
    public Object clone() throws CloneNotSupportedException {
        return super.clone();
    }
}
\end{lstlisting}

具体用法:(\dm{clone}方法抛出的异常为受查异常,需显示处理)
\begin{lstlisting}
A a = new A();
Object o;
A b;
try {
    o = a.clone();
    b = (A)(o);
}catch(Exception e){
    System.out.println(e.getMessage());
}
\end{lstlisting}
该拷贝为浅拷贝,如果原型对象成员变量是值类型,将复制一份给克隆对象,如果原型对象的成员变量是引用类型,则将该引用复制一份给克隆对象。改变某一对象中的引用类型数据,另一个也会随之改变。(对于String、Integer这些不可变对象除外。)

深拷贝会创造一个一模一样的对象,新对象与原对象不共享内存。

实现方法:使用序列化先将对象写入流中,然后再从流中读入到新对象中。

\subsection{\dm{ByteArrayOutputStream}}

该流不会写入文件中,而是在虚拟机上,也就是在内存上。

\begin{lstlisting}[title=深拷贝]
class B implements Cloneable, Serializable {
    //其他的属性和方法
    @Override
    public Object clone() throws CloneNotSupportedException {
        Object o = null;
        ByteArrayOutputStream baos = null;
        ObjectOutputStream oos = null;
        ByteArrayInputStream bais = null;
        ObjectInputStream ois = null;
        try {
            baos = new ByteArrayOutputStream();
            oos = new ObjectOutputStream(baos);
            oos.writeObject(this);//把文件写入baos
            byte[] bs = baos.toByteArray();//将其转为字节数组,放的是当前对象的全部字节

            bais = new ByteArrayInputStream(bs);
            ois = new ObjectInputStream(bais);
            o = ois.readObject();

        } catch (Exception e) {
            System.out.println(e.getMessage());
        } finally {
            try {
                oos.close();
                baos.close();
                ois.close();
                bais.close();
            } catch (Exception e1) {
                System.out.println(e1.getMessage());
            }
        }
        return o;
    }
}
\end{lstlisting}
\section{字符流}
使用字符流可以解决字符的编码问题。字符的编码方式和解码方式不统一,可能会造成乱码问题。

把字符转为字节码称为编码,把字节码转为字符称为解码。

常见的编码方式:ASCII,ISO-8859-1(西欧),GB2312(简体中文,字符集较小),GBK(简体中文,字符集多于GB2312,兼容GB2312),Big5(繁体中文),Unicode(全球统一编码,变种UTF-16一个字符16位,两个字节;UTF-8(变长,最常见的))

无论采用何种编码,其都向上兼容ASCII码。

\subsection{字符流的输入输出流}
\dm{Writer}和\dm{Reader}是字符流的父类,属于抽象类。

\dm{FileReader}和\dm{FileWriter}是文件字符流。

\dm{BufferedReader}和\dm{BufferedWriter}是过滤流。缓冲

\begin{tips}
\dm{BufferedReader}有一个方法\dm{readline()},每次从文件中读取一行,读到换行符为止,返回值为字符串。当返回为null时,表示文件读取结束。
\end{tips}

\dm{PrinterWriter}缓冲  过滤流。

\begin{tips}
\dm{PrinterWriter}有一个方法\dm{println()},每次向文件中写入一行。
\end{tips}

\subsection{字节流转字符流}
\dm{OutputStreamWriter}(桥转换类)将字节流转换为字符流。\dm{InputStreamWriter}
\begin{lstlisting}
OutputStream fos = new FileOutputStream("a.txt");
Writer osw = new OutputStreamWriter(fos);//使用字节输出流构造出字符输出流
\end{lstlisting}

\begin{tips}
在使用桥转换功能时,可以指定编码解码方式。
\begin{lstlisting}
OutputStream fos = new FileOutputStream("a.txt");
Writer osw = new OutputStreamWriter(fos,"GBK");//指定编码方式为GBK
\end{lstlisting}
\hphantom{~}
\end{tips}

\section{\dm{file}类}
IO流:对文件中的内容进行操作

File类:对文件自身进行操作,例如创建文件,删除文件,文件的重新命名,文件的权限设置等等。不能访问文件内容。

File对象可以代表磁盘上的文件或目录。

\begin{lstlisting}
File f = new File("a.txt");//只是创建了一个File对象,并没有在磁盘上创建文件
try{
    f.createNewFile();//把上边的File对象写入磁盘
}catch(Exception e){
    System.out.println(e.getMessage());
}
f.delete();//把文件从磁盘上删除(彻底删除,不会在回收站)

//在磁盘上创建目录
f.mkdir();
//删掉目录
f.delete();//如果目录不为空,则无法删掉,只能删掉空目录

//重命名
File f1 = new File("a1.txt");
f.renameTo(f1);//该方法的参数是File对象,成功返回true,失败返回false

//设置文件为只读
f.setReadOnly();

//获取文件名:含扩展名,文件夹名
f.getName();

//判断一个文件或目录是否存在
f.exists();

//获得绝对路径
f.getAbsolutePath();

//若f表示的是目录,则获取当前文件下所有的文件、文件夹,返回值类型为File[]
f.listFiles();

//判断File对象是否为文件,返回值为false是可能文件不存在,要注意这点
f.isFile();

//判断File对象是否为目录,返回值为false是可能目录不存在,要注意这点
f.isDirectory();
\end{lstlisting}

\begin{lstlisting}[title=删除目录]
//若某个目录下有子文件,需使用递归的方法删除
public static void delDir(File dir){
    File[] fs = dir.listFiles();
    for (File file: fs) {
        if (file.isFile()){//file是文件则直接删掉
            file.delete();
        }
        if (file.isDirectory()){//file是目录则删掉目录
            delDir(file);
        }
    }
    dir.delete();//将dir目录删除掉
}
\end{lstlisting}

\subsection{文件过滤器}
\dm{public File[] listFiles(FileFilter filter);}

\dm{FileFilter}是一个接口,需要自己实现。
\begin{lstlisting}
File dir = new File("E:/corejava");
File[] fs = dir.listFiles(new FileFilter(){
    @Override
    public boolean accept(File pathname){
        if(pathname.isFile()) return true;//是文件就返回true
        return false;//如果希望结果保存在fs中,返回true,否则返回false
    }
});
\end{lstlisting}

\begin{lstlisting}[title=列出目录所有的.java文件]
public static void listJavaFiles(File dir){
    File[] ret = dir.listFiles(new FileFilter() {
        @Override
        public boolean accept(File pathname) {
            if (pathname.isDirectory()) return true;
            if (pathname.isFile()){
                String name = pathname.getName();
                if (name.endsWith(".java")){
                    return true;
                }
            }
            return false;
        }
    });
    for (File f: ret) {
        if (f.isFile()) System.out.println(f.getAbsolutePath());
        if (f.isDirectory()) listJavaFiles(f);
    }
}
\end{lstlisting}
\chapter{NIO}
\dm{Buffer}:缓冲区
\begin{lstlisting}
import java.nio.*;

ByteBuffer buffer = ByteBuffer.allocate(20);//创建了一个容量为20字节的Buffer缓冲对象

byte[] bs = new byte[20];
ByteBuffer buffer2 = ByteBuffer.wrap(bs);//创建方式2
\end{lstlisting}
\dm{Channel}:通道
\begin{lstlisting}[title=写数据]
FileOutputStream fos = new FileOutputStream("a.txt");//文件输出流
FileChannel channel = fos.getChannel();//建立管道

ByteBuffer buffer = ByteBuffer.wrap("hello".getBytes());//将需要写的数据写入缓冲区
channel.write(buffer);//用管道把缓冲区写入文件
channel.close();
\end{lstlisting}

\begin{lstlisting}[title=读数据]
FileInputStream fis = new FileInputStream("a.txt");//文件输入流
FileChannel channel = fis.getChannel();//建立管道

ByteBuffer buffer = ByteBuffer.allocate(20);
while(true){
    int len = channel.read(buffer);//将需要读的数据读入缓冲区
    if(len == -1){
        break;
    }
    buffer.flip();//buffer中有三个指针,position,limits,capacity,该方法是将position置为0,即从写模式切换为读模式
    while(buffer.hasRemaining()){//如果buffer中还有元素未读,就继续读
        //从position读到limits
        //buffer.get();//还有其他get方法,例如getLong()
        System.out.println((char)(buffer.get()));
    }
    buffer.clear();//把limits,position置为0,即从读模式切换为写模式
}

buffer.close();
channel.close();
fis.close();
\end{lstlisting}
\begin{lstlisting}[title=字节流转字符流]
//方式1
ByteBuffer bb = ByteBuffer.wrap("你好".getBytes());

//方式2
Charset cs = Charset.forName("GBK");
ByteBuffer bb2 = cs.encode("你好");

CharBuffer cb = cs.decode(bb2);
\end{lstlisting}
\subsection{文件拷贝}
\begin{lstlisting}[title=文件拷贝]
//文件拷贝3
//该算法效率低于上边的拷贝2
public static void fileCopy3(String src, String dst) {
    FileInputStream fis = null;
    FileOutputStream fos = null;
    ByteBuffer buffer = ByteBuffer.allocate(1024);
    FileChannel channel1 = null;
    FileChannel channel2 = null;
    try {
        fis = new FileInputStream(src);
        fos = new FileOutputStream(dst);
        channel1 = fis.getChannel();
        channel2 = fos.getChannel();
        while (true) {
            int a = channel1.read(buffer);
            if (a == -1) {
                break;
            }
            buffer.flip();//写模式变为读模式
            channel2.write(buffer);
            buffer.clear();//读模式变为写模式
        }
    } catch (Exception e) {
        System.out.println(e.getMessage());
    } finally {
        try {
            channel2.close();
            channel1.close();
            fos.close();
            fis.close();
        } catch (Exception e1) {
            System.out.println(e1.getMessage());
        }
    }
}
\end{lstlisting}
\begin{lstlisting}
//文件拷贝4
public static void fileCopy4(String src, String dst) {
    FileInputStream fis = null;
    FileOutputStream fos = null;
    ByteBuffer buffer = ByteBuffer.allocate(1024);
    FileChannel channel1 = null;
    FileChannel channel2 = null;
    try {
        fis = new FileInputStream(src);
        fos = new FileOutputStream(dst);
        channel1 = fis.getChannel();
        channel2 = fos.getChannel();
        MappedByteBuffer map = channel1.map(FileChannel.MapMode.READ_ONLY, 0, channel1.size());//虚拟内存映射
        channel2.write(buffer);
    } catch (Exception e) {
        System.out.println(e.getMessage());
    } finally {
        try {
            channel2.close();
            channel1.close();
            fos.close();
            fis.close();
        } catch (Exception e1) {
            System.out.println(e1.getMessage());
        }
    }
}
\end{lstlisting}

\begin{lstlisting}
//文件拷贝5
//性能最高
public static void fileCopy5(String src, String dst) {
    FileInputStream fis = null;
    FileOutputStream fos = null;
    ByteBuffer buffer = ByteBuffer.allocate(1024);
    FileChannel channel1 = null;
    FileChannel channel2 = null;
    try {
        fis = new FileInputStream(src);
        fos = new FileOutputStream(dst);
        channel1 = fis.getChannel();
        channel2 = fos.getChannel();
        channel1.transferTo(0,channel1.size(),channel2);
    } catch (Exception e) {
        System.out.println(e.getMessage());
    } finally {
        try {
            channel2.close();
            channel1.close();
            fos.close();
            fis.close();
        } catch (Exception e1) {
            System.out.println(e1.getMessage());
        }
    }
}
\end{lstlisting}
\end{document} 