\documentclass[a4paper]{report}
\usepackage[space,fancyhdr,fntef]{ctexcap}
\usepackage{fontspec}
\fontspec{宋体}
\setmainfont{Times New Roman}
%\fontsize{50pt}{50pt}\selectfont
\renewcommand{\rmdefault}{ptm}
\usepackage[namelimits,sumlimits,nointlimits]{amsmath}
\usepackage[text={169mm,250mm},bottom=20mm,top=25mm,left=25mm,right=15mm,centering]{geometry}
\usepackage{color}
\usepackage{CJKfntef}%下划线宏包160页
\usepackage{xcolor}
\usepackage{arydshln}%234页,虚线表格宏包
\pagestyle{fancy} \fancyhf{}
\fancyhead[OC]{\color{gray}\rightmark}

\fancyhead[EC]{\color{gray}\leftmark}
\fancyfoot[C]{\color{gray}\thepage}
\renewcommand{\headrule}{\color{gray}\hrule width\headwidth}
%\renewcommand{\footrulewidth}{0.4pt}%改为0pt即可去掉页脚上面的横线
%\usepackage{parskip}
%\usepackage{indentfirst}
\usepackage{graphicx}%插图宏包,参见手册318页
\definecolor{dkgreen}{RGB}{106,135,89}
\definecolor{dkblue}{RGB}{103,150,163}
\definecolor{wgray}{RGB}{248,248,248}
\definecolor{WGRAY}{RGB}{248,248,248}
\usepackage{listings}
\lstset{language=Java,
backgroundcolor=\color{wgray},
rulesepcolor=\color{red!20!green!20!blue!20},%代码块边框为淡青色
%lablestep=1,
%lablesep=5pt,
%lablestyle=\tiny,
%tablesize=4,
%captionpos=b,
basicstyle=\ttfamily\small,
keywordstyle=\color{orange},
commentstyle=\color{gray},
stringstyle=\color{dkgreen},
numberstyle=\tiny,
numbersep=8pt,
frame=single,%topline.bottomline,lines,single,leftline
identifierstyle=\color{dkblue},
numbers=left,
stepnumber=1,
xleftmargin=2em,xrightmargin=2em, aboveskip=1em,
breaklines=true
}
\usepackage[xetex,colorlinks]{hyperref}%394页  \href{网址}{文本}
\hypersetup{urlcolor=blue}
%\linebreak[2]%换行,152页
\usepackage{fancybox}%盒子宏包55页
\setcounter{secnumdepth}{4}
\CTEXoptions[contentsname={目\hspace{15pt}录}]
\CTEXsetup[beforeskip={-40pt},afterskip={20pt plus 2pt minus 2pt}]{chapter}

%目录设置
\usepackage{titletoc}
\usepackage[toc]{multitoc}
\titlecontents{chapter}[4em]{\addvspace{2.3mm}\bf}{\contentslabel{4.0em}}{}{\titlerule*[5pt]{$\cdot$}\contentspage}
\titlecontents{section}[4em]{}{\contentslabel{2.5em}}{}{\titlerule*[5pt]{$\cdot$}\contentspage}
\titlecontents{subsection}[7.2em]{}{\contentslabel{3.3em}}{}{\titlerule*[5pt]{$\cdot$}\contentspage}
\usepackage{fancyvrb}%75页抄录宏包
\begin{document}
\flushbottom%版心底部对齐
\newcommand{\dm}[1]{\colorbox{wgray}{\lstinline`#1`}}
\newcommand{\myroman}[1]{\uppercase\expandafter{\romannumeral#1}}
\newcounter{num}[section] \renewcommand{\thenum}{\arabic{num}.} \newcommand{\num}{\refstepcounter{num}\text{\thenum}}

\newenvironment{tips}{\kaishu\zihao{-6}\color{blue}{\noindent\rule[-3pt]{\textwidth}{0.5pt}\par \em \noindent {\zihao{-5} \textcolor[rgb]{1.00,0.00,0.00}{Tips}}}\par}{\\ \rule[3mm]{\textwidth}{0.5pt}\par}

\newenvironment{zhengming}{\kaishu\zihao{-5}\color{blue}{\noindent\em 证明:}\par}{\hfill $\diamondsuit$\par}

\tableofcontents
\pagenumbering{Roman}%设置目录页码
\clearpage
\pagenumbering{arabic}%设置正文页码
\chapter{JavaSE}
\href{https://docs.oracle.com/javase/8/docs/api/index.html}{Java官方文档}

快捷键:

1.ctrl+鼠标左键:把光标放在\dm{String}上,打开对应的源码
\section{环境}
\subsection{基础概念}
\begin{itemize}
\itemsep=0pt \parskip =0pt
  \item JDK:Java开发工具包,开发Java代码必备
  \item JRE:Java运行时环境,运行Java程序必备
  \item JVM:Java虚拟机
\end{itemize}
三者的关系:JDK中包含了JRE,JRE中又有JVM。
\subsection[编译Java程序]{编译运行Java程序(命令行)}
\begin{itemize}
\itemsep=0pt \parskip =0pt
  \item javac:编译,把\dm{.java}文件变成\dm{.class}二进制字节码文件
  \item java:让JVM解释执行字节码文件
\end{itemize}

\subsection{集成开发环境}
使用IDEA集成开发环境进行Java程序的开发

\section{变量和类型}
\subsection{变量}
变量:表示程序运行时可以改变的量,创建一个变量,就会为其分配一定的内存空间
\subsection{常量}
\begin{itemize}
\itemsep=0pt \parskip =0pt
  \item 字面值常量:例如$1,1.0,1.0f,"hello",true,false$等。
  \item \dm{final}关键字修饰的量,在程序运行中无法修改其值。
\end{itemize}
\subsection{内置类型}
\begin{itemize}
\itemsep=0pt \parskip =0pt
  \item 整数:\dm{byte},\dm{short},\dm{int},\dm{long}
  \item 浮点数:\dm{float},\dm{double},不能使用$=$直接判断两个浮点数是否相等。
  \item 字符:\dm{char},Java中的char占两个字节,使用unicode编码。
  \item 布尔:\dm{boolean}
\end{itemize}
内置类型都有对应的包装类来描述他们。
\subsection{引用类型}
\begin{itemize}
\itemsep=0pt \parskip =0pt
  \item 字符串\dm{String}
  \item 数组
  \item 类
\end{itemize}

引用相当于是一个低配指针,没有指针加减的一些功能。

\subsection{类型转换}
隐式类型转换:把一个表示范围小的类型赋值给表示范围大的类型

强制类型转换:用加括号的方式来进行强制类型转换。要注意部分类型之间无法进行强制类型转换,如布尔和整型,如果强制转换,则会出错。

类型提升:把不同类型的数值放一起进行计算,会发生类型提升,\dm{byte}、\dm{short}类型进行计算时,通常先提升为\dm{int}类型进行计算。

\subsection{变量的作用域}
局部变量:作用域为当前代码块。

成员变量:取决于\textbf{访问权限控制}字符。\dm{public}、\dm{private}、\dm{protected}、\dm{default}(就是空白)
\subsection{变量的命名规范}

1.必须由数字、字母、下划线、$\$$组成。

2.变量名尽量有意义,采用驼峰命名法。(变量名小驼峰,类名大驼峰)

\section{运算符}
1.算数运算符:\dm{+}、\dm{-}、\dm{*}、\dm{/}、\dm{\%}。

2.关系运算符:\dm{<}、\dm{<=}、\dm{==}、\dm{>}、\dm{>=}、\dm{!=},关系运算符的返回值一定是\dm{boolean}类型。

3.逻辑运算:\dm{&&}、\dm{||}、\dm{!},逻辑运算符支持\textcolor[rgb]{1.00,0.00,0.00}{短路求值}。

\dm{&}和\dm{|}的操作数为\dm{boolean}类型,也表示逻辑运算,但不支持短路求值。

4.位运算:$\&$、$|$操作数是整数类型,$\sim$表示取反,$\^$表示异或。

5.移位运算:\dm{<<}、\dm{>>}(算术右移,左侧补符号位)、\dm{>>>}(逻辑右移,左侧补0)。左移1位,相当于乘2,算数右移1位,相当于除2.

6.条件运算:\dm{? :}

7.运算符的优先级:不知道优先级时加括号。

\section{注释}
3种。
\section{关键字}

\section{分支和循环语句}
\subsection{分支语句}
1.\dm{if}

2.\dm{switch}
\subsection{循环语句}
1.\dm{while}语句

2.\dm{for}语句

3.\dm{do...while}语句:至少执行一次

4.\dm{for...each}

5.\dm{continue}:跳过本次循环,直接进入下一循环

6.\dm{break}:直接结束循环。
\section{输入和输出}
\subsection{输出}
1.\dm{System.out.println()}:打印换行

2.\dm{System.out.printf()}:格式化输出

3.\dm{System.out.print()}:不格式化输出不打印换行

\subsection{输入}
1.\dm{System.in}:需要手动处理异常

2.借用\dm{Scanner}类来进行输入

\section{方法}
1.基本语法:方法必须在类中,不能单独存在

修饰限定符,返回值,方法名(驼峰命名),参数列表,方法体(只有在调用的时候才会执行方法体)

2.方法调用:方法名+(实参列表),代码要进入方法体内部执行,并且完成参数传递过程,方法执行完毕,回到调用位置继续执行。

3.方法重载:方法名相同,方法在同一个作用域中,方法的参数个数不同,参数类型不同。但是方法返回值的类型不影响重载。(\textcolor[rgb]{1.00,0.00,0.00}{重写}和\textcolor[rgb]{1.00,0.00,0.00}{重载}是两个完全不一样的概念)

\section{数组}

概念:批量创建一组相同类型的变量

创建方式:
\begin{itemize}
\itemsep=0pt \parskip =0pt
\item \dm{int[] a = {1, 2, 3, 4};}
\item \dm{int[] a = new int[]{1, 2, 3, 4};}
\item \dm{int[] a = new int[num];}\dm{num}为数组的大小。
\end{itemize}

使用:
\begin{itemize}
\itemsep=0pt \parskip =0pt
\item \dm{arr.length}用于求数组长度
\item \dm{arr[i]},数组的下标是从0开始的,如果超出范围,则会抛出数组下标越界异常
\end{itemize}

引用:

\section{面向对象}
\subsection{类和对象的概念}
类相当于是图纸,对象是根据图纸造出来的房子。类相当于C语言中的结构体,而对象则是该自定义类型创建出来的变量。

\subsection{基本语法}
类的创建:
\begin{itemize}
\itemsep=0pt \parskip =0pt
\item 修饰符:\dm{public}
\item \dm{class}
\item 类名:\dm{public}修饰的类名必须和文件名一致,通常以大驼峰的形式命名。
\item 类体:用一对大括号括起来,里边包括成员变量和成员函数/方法
\end{itemize}

类的实例化(对象的创建):使用\dm{new}关键字

\subsection{\dm{this}关键字}
\dm{this}关键字:
\begin{itemize}
\itemsep=0pt \parskip =0pt
\item 获取到当前对象的引用,即指向当前对象的引用。
\item \dm{this}的类型就是当前类的类型。
\item \dm{this}这个引用的指向不能修改。
\item \dm{this}也可以用来调用方法
\item \dm{this}不能是\dm{null}
\end{itemize}

当成员的名字和参数名字相同时,通过\dm{this.}的方式显示的区分出方法的参数名和成员的名字。
\subsection{对象的初始化}
1.默认值初始化:整数浮点数都初始化为0,布尔类型初始化为\dm{false},引用类型初始化为空指针

2.就地初始化:在创建成员的时候直接使用\dm{=}来赋值进行初始化。

3.代码块初始化:在类里面用大括号进行初始化

4.构造方法。(快捷键:alt+ins)
\begin{itemize}
\itemsep=0pt \parskip =0pt
  \item 构造方法的方法名和类名一致。
  \item 构造方法不需要写返回值的类型,内部也不需要写\dm{return}语句。
  \item 构造方法不需要显示调用,\dm{new}的时候自动被调用
  \item 构造方法支持重载
\end{itemize}

\subsection{\dm{toString}}
用于打印对象的信息。
\begin{lstlisting}[title=\dm{toString}方法]
@Override
public String toString(){
    return this.name + "," + this.gender//假设这个类里面有name和gender两个成员变量
}
\end{lstlisting}

也可以用alt+ins快捷键直接用.

\section{包}
包就相当于是目录,当代码中文件太多,就需要放在不同的目录中,即放在不同的包中。

导入包中的类:\dm{import 包名}

\subsection{常见的系统包}
1.\dm{java.lang}:系统常用基础类\dm{String}、\dm{Object},此包从JDK1.1后自动导入。

2.\dm{java.lang.reflect:java}:反射编程包

3.\dm{java.net}:进行网络编程开发包

4.\dm{java.sql}:进行数据库开发的支持包

5.\dm{java.util}:是Java提供的工具程序包。
\subsection{\dm{static}关键字}
1.类中的某个成员加上\dm{static},说明这个成员是一个类属性/类方法。,如果没有\dm{static},成员是一个实例属性/实例方法。

2.类属性可以用类名访问。如:\dm{Cat.n}(这里\dm{Cat}是个类名,\dm{n}是类里边由\dm{static}关键字修饰的成员变量)。即不需要创建实例来通过类名访问类成员变量。

3.修饰方法:静态方法。通过类名来访问,即调用静态方法不需要创建实例。\dm{this}关键字是只当前对象的引用,调用静态方法没有创建实例,所以在静态方法中不能使用\dm{this}。(静态方法和实例无关,只和类有关)同理也无法在\dm{static}方法中访问非\dm{static}的变量和方法。

4.修饰代码块:例如
\begin{lstlisting}[title=\dm{static}修饰代码块]
static {
    //这个加上static的代码块叫做“静态代码块”
    //静态代码块只在类加载的时候执行一次
    //一般用来初始化静态成员
    //类加载始终是在创建实例之前
    //static修饰的代码块始终是在普通的代码块之前执行的
}
\end{lstlisting}

\subsection{静态导入包}
\dm{import static 包名}

\section{访问限定符}
面向对象的特性包括:类和对象,抽象,封装,继承,组合,多态,反射/自省等。其中封装、继承、多态最具有代表性。
\subsection{\dm{public}}
修饰的成员可以被外部的类随意访问。

\subsection{\dm{private}}
修饰的成员只能在自己类的内部使用

\subsection{\dm{protected}}
修饰的成员

\subsection{\dm{default}}
即成员前不加任何修饰,这种成员只能在当前的包里被使用。(包级访问权限)

\section{内部类}
比较少用,仅作了解。

把类的定义写在另一个类中。

1.普通的内部类/成员内部类

2.静态内部类(内部类前用\dm{static}修饰)

静态内部类不依赖外部的\dm{this},可以随意创建

3.匿名内部类(相对比较常用)

\begin{lstlisting}[title=匿名内部类]
//假设当前包里有个A类。
//此时创建了一个匿名内部类,这个类没有名字
//这个类是A类的子类(继承自A类)
A a = new A(){
    //定义相关属性和方法
}
\end{lstlisting}

4.局部内部类

把类定义到方法里。

\section{栈堆}
JVM中的内存区除了堆和栈之外,还有方法区,方法区存的是“类相关的信息”。

对于属性来说,如果是\textbf{实例属性},其信息跟着实例走,如果是\textbf{类属性},其信息在方法区中,对于方法来说,无论是否有\dm{static}修饰,对应的内容都是在方法区中。

\section{\dm{String}类}
\subsection{创建字符串}
%\begin{lstlisting}[title=创建字符串]
%//方式1
%String str = "Hello";
%//方式2
%String str = new String("Hello");
%//方式3
%char[] array = {'a', 'b', 'c'};
%String str = new String(array);
%\end{lstlisting}
\begin{lstlisting}[title=创建字符串]
//方式1
String str = "Hello";
//方式2
String str = new String("Hello");
//方式3
char[] array = {'a', 'b', 'c'};
String str = new String(array);
\end{lstlisting}
Java中的字符串(\dm{String})和字符数组(\dm{char[]})之间没有关联关系

\subsection{字符串比较}
Java中如果针对引用类型使用\dm{==},此时比较的是两个引用的地址是否相等。

对于:
\begin{lstlisting}[title=字符串比较1]
String str1 = "Hello";
String str2 = "Hello";
\end{lstlisting}
上边的\dm{str1}和\dm{str2}创建的的地址一样即\dm{str1 == str2}为\dm{true},但是如果通过上边的方式2、3创建的,则不相等。

Java中字符串常量会被保存到字符串常量池中,此时字符串只需要保存一份即可。


比较字符串的内容是否相等应使用\dm{.equals()}方法。例如:
\begin{lstlisting}[title=字符串比较2]
String str1 = "Hello";
String str2 = "Hello";
System.out.println(str1.equals(str2));
\end{lstlisting}

如果字符串变量和字符串常量比较,应按如下格式书写:
\begin{lstlisting}[title=字符串比较3]
String str1 = "Hello";

//不建议写成这种方式,一旦str1是null,此时就会抛出空指针异常
if(str1.equals("Hello")){
    //代码块
}
//建议写成如下方式,如果str1是null,不会抛出异常,而是返回false
if("Hello".equals(str1)){
    //代码块
}
\end{lstlisting}

\subsection{字符串常量池}
池:计算机中一个非常重要的术语。例如内存池,线程池,进程池,数据库连接池,对象池……等。

池的目的就是为了降低开销,提高效率。本质是把频繁使用的东西保存好,以备用到的时候随时就能用。

Java会把一些字符串常量放到内置的“字符串常量池”中。

可以使用\dm{String}中的\dm{intern()}的方法手动把\dm{String}对象加入到字符串的常量池中。即:
\begin{lstlisting}[title=加入字符串到常量池]
String str1 = new String("Hello").intern();
\end{lstlisting}
此段代码的含义是,\dm{new}了一个\dm{String}对象,该\dm{String}对象的内容是\dm{"hello"},然后调用\dm{intern}方法,该方法拿着当前的字符串去字符串常量池中找,看当前这个内容是否存在于池中。

如果存在,直接返回该池中字符串对应的地址,如果不存在,则把当前的字符串内容加到常量池中,然后返回池中的地址。
\subsection{字符串不可变}
%在第12集的2小时17分处
Java的\dm{String}是不可变对象(对象本身不能修改)。

\dm{final}修饰的是常量,如果修饰的是一个引用类型,表示的是该引用的指向(引用中存的地址不能改),若修饰类,表示这个类不能被其他类继承。

在\dm{String}的实现源码中,是用\dm{private}修饰的字符数组来存放字符串内容的,所以无法在\dm{String}类的外部通过[]的形式来获取或修改该字符数组的内容。

Java的\dm{String}设计成不可变的原因
\begin{itemize}
\itemsep=0pt \parskip =0pt
  \item 方便放入池中,如果是可变的对象,一旦池中的内容发生改变,就会影响到所有引用这个池对象的结果
  \item 对象内容不可变,则对象的hashCode也不可变。方便和hash表这样的结构配合使用
  \item 对象不可变,线程安全更有保证
\end{itemize}
Java中为了方便修改,提供了\dm{StringBuilder}和\dm{StringBuffer}这样的类。

通过“反射/自省”的方式修改字符串内容,例如
\begin{lstlisting}[title=修改字符串]
String str1 = "Hello";
//通过反射的方式修改"Hello"的内容
//反射是特殊手段,而不是常规手段

//1.获取到String的类对象
//2.根据“value”这个字段名字,在类对象中拿到对应的字段(仍然是图纸的一部分,相当于是局部放大图)
Field valueField = String.class.getDeclaredField("value");
//让value这个private的成员也能被访问
valueFiled.setAccessible(true);
//3.根据图纸,把str这个对象拆开,取出里面的零件
char[] value = (char[])valueFiled.get(str);
//4.修改零件的内容
value[value.length] = 'a';
\end{lstlisting}

C++的\dm{std::string}就是可变对象
\begin{itemize}
\itemsep=0pt \parskip =0pt
  \item 可变对象方便修改(修改起来比较高效)
\end{itemize}
C++为了高效的保存字符串,也有类似的“池”机制,由于是可变对象,涉及到“写时拷贝”机制。

\subsection{字符串、字符、字节}
\subsubsection{字符转字符串}
使用一个数组,构造字符串。例如:
\begin{lstlisting}[title=字符转字符串]
char s = {'a', 'b', 'c'};
String str = new String(s);
\end{lstlisting}
\subsubsection{字符串转字符}
使用\dm{charAt}来获取到指定下标位置的字符。
\begin{lstlisting}[title=字符串转字符]
String str = "abc";
System.out.println(str.charAt(0));
//使用.length()方法获取到字符串长度,而数组的长度.length是一个属性
System.out.println(str.length());
\end{lstlisting}

\dm{toCharArray}方法,该方法相当于是在内部创建了一个新的字符数组并返回,修改这个返回值并不会影响到原字符串的内容。
\subsubsection{字节转字符串}
使用一个字节数组来构造字符串(要求字节数组中保存的内容符合Java字符的编码方式)。


\subsubsection{字符串转字节}
使用\dm{getBytes}方法完成,在网络编程中经常会用到。
\begin{lstlisting}[title=字符串转字节]
String str = "abc";
str.getBytes();//可以把当前字符串的内容转换到字节数组中
\end{lstlisting}
\subsection{字符串常见操作}
\subsubsection{字符串的比较}
1.比较相等:\dm{equals}是区分大小写的比较,\dm{equalsIgnore}是不区分大小写的比较

2.比较大小:(按字典序来比较大小)\dm{compareTo()}

\begin{lstlisting}[title=字符串比较大小]
String str1 = "abc";
String str2 = "abcd";
int result = str1.copmareTo(str2);
\end{lstlisting}
\begin{itemize}
\itemsep=0pt \parskip =0pt
  \item 如果str1大于str2,返回一个正数
  \item 如果str1等于str2,返回0
  \item 如果str1小于str2,返回一个负数
\end{itemize}

同理,\dm{compareToIgnoreCase}为忽略大小写的比较。
\subsubsection{字符串查找}
\begin{lstlisting}
//判断一个字符串是
public boolean contains(CharSequence s)
//从头开始查找指定字符串的位置,查到了返回位置的开始索引,如果查不到返回-1
public int indexOf(String str)
//从指定位置开始查找子字符串的位置
public int indexOf(String str, int fromIndex)
//从后向前查找子字符串的位置
public int lastIndexOf(String str)
//由指定位置从后向前查找
public int lastIndexOf(String str, int fromIndex)
//判断是否以指定字符串开头
public boolean startsWith(String prefix)
//从指定位置判断是否以指定字符串开头
public boolean startsWith(String prefix)
//判断是否以指定字符串结尾
public boolean endsWith(String suffix)
\end{lstlisting}

开头结尾函数用法:
\begin{itemize}
\itemsep=0pt \parskip =0pt
  \item 判断某个链接的协议类型,用\dm{startsWith}
  \item 判断某个文件的类型,会用\dm{endsWith}判断其扩展名
\end{itemize}
\subsubsection{字符串替换}
\begin{lstlisting}
//替换所有指定内容
public String replaceAll(String regex, String replacement)
//替换首个内容
public String replaceFirst(String regex, String replacement)
\end{lstlisting}
由于字符串是不可变对象,替换不会修改原字符串,而是产生一个新的字符串。

\begin{tips}
\dm{String regex}:正则表达式

正则表达式使用一些特殊的字符来描述某些字符串的筛选标准。正则表达式也是一个字符串。
\end{tips}

\subsubsection{字符串拆分}
将一个字符串按照指定的分隔符划分为若干个子字符串
\begin{lstlisting}
//将字符串全部拆分
public String[] split(String regex)
//将字符串部分拆分,该数组长度就是limit极限
public String[] split(String regex, int limit)
\end{lstlisting}

\textbf{注意事项}:
\begin{itemize}
\itemsep=0pt \parskip =0pt
  \item 字符$"|","*","+"$都得加上转义字符,前面加上$"\backslash"$。
  \item 如果是$"."$,那么就得写成$"\backslash\backslash ."$。
  \item 如果一个字符串有多个分隔符,可以用$"|"$作为连字符。
\end{itemize}

\subsubsection{字符串截取}
从一个完整的字符串中截取出部分内容
\begin{lstlisting}
//从指定索引截取到结尾
public String substring(int beginIndex)
//截取部分内容
public String substring(int beginIndex,int endIndex)
\end{lstlisting}
\textbf{注意事项}:
\begin{itemize}
\itemsep=0pt \parskip =0pt
  \item 索引是从0开始的。
  \item \dm{substring(int beginIndex,int endIndex)},包含下标为\dm{beginIndex}的字符,不包含\dm{endIndex}的字符。
\end{itemize}
\subsubsection{其他操作方法}
\begin{lstlisting}
//去掉字符串的左右空白符(空格,换行,回车,制表符等),保留中间空格
public String trim()
//字符串转大写
public String toUpperCase()
//字符串转小写
public String toLowerCase()
//字符串入池
public native String intern()
//字符串链接,等同于+,不入池
public String concat(String str)
//求字符串长度
public int length()
//判断是否为空字符串"",但不是null,而是长度为0
public boolean isEmpty()
\end{lstlisting}
\dm{native}关键字修饰的方法称为本地方法,表明此方法不是由Java实现的,而是由实现JDK的语言实现的。

\dm{String}类并没有提供首字母大写操作,需自己实现。

\begin{Verbatim}[frame=single,numbersep=5pt,xleftmargin=1.5em,xrightmargin=1.5em]
public static String firstUpper(String str){
    if ("".equals(str) || str == null){
        return str;
    }
    if (str.length() > 1){
        return str.substring(0,1).toUpperCase() + str.substring(1);
    }
    return str.toUpperCase();
}
\end{Verbatim}
%\begin{lstlisting}
%public static String firstUpper(String str){
%    if ("".equals(str) || str == null){
%        return str;
%    }
%    if (str.length() > 1){}
%        return str.substring(0,1).toUpperCase() + str.substring(1);
%    }
%    return str.toUpperCase();
%}
%\end{lstlisting}

\subsection{\dm{StringBuffer}和\dm{StringBuilder}}
如果需要可变版本的\dm{String},就需要\dm{StringBuffer}或\dm{StringBuilder},这两个类的用法基本一致,下边以其中一个为例。
\begin{lstlisting}[title=\dm{StringBuilder}用法]
//1.append字符串的追加,等同于String的+=
//Stringde +=会产生新的String对象,如果在循环中使用则极其低效
StringBuilder stringBuilder = new StringBuilder("hello");
for(int i = 0; i < 100; i++){
    stringBuilder.append(i);
}
//字符串反转
stringBuffer.rereverse();
System.out.println(stringBuilder.toString());
//字符串删除
stringBuilder.delete(2,4);//从2删除到4(包含2不包含4)
System.out.println(stringBuilder.toString());
//插入数据
stringBuilder.insert(0,"word");在位置0插入字符串word
\end{lstlisting}

\dm{String}、\dm{StringBuffer}和\dm{StringBuilder}的区别
\begin{itemize}
\itemsep=0pt \parskip =0pt
  \item \dm{String}的内容不可以修改,\dm{StringBuffer}与\dm{StringBuilder}的内容可以修改
  \item \dm{StringBuffer}与\dm{StringBuilder}大部分功能相似。
  \item \dm{StringBuffer}采用同步处理,属于线程安全操作,\dm{StringBuilder}未采用同步处理,属于线程不安全操作。
\end{itemize}

\dm{synchronized}关键字:同步,一个方法加上该关键字,很可能就是线程安全的。
\section{面向对象}
主要是关于继承、组合、多态、抽象类、接口。
\subsection{继承}
目的就是为了让代码能够很好的被重复利用,为了把类进行代码重用。

两个核心概念:父类(被继承)(又叫基类、超类)、子类(继承)(又叫派生类)

\dm{extends}关键字
\begin{lstlisting}[title=继承语法]
class 子类 extends 父类{

}
\end{lstlisting}
\begin{itemize}
\itemsep=0pt \parskip =0pt
  \item 继承也有一层“扩展”的意思,即在类保持现有功能的前提下,加入新的功能。
  \item 使用\dm{extends}指定父类。
  \item 子类会继承父类的所有的属性和方法
  \item 对于父类的\dm{private}的字段和方法,子类中是无法访问的
  \item 子类的实例中,也包含着父类的实例。可以使用\dm{super}关键字得到父类实例的引用
\end{itemize}

继承就是为了代码重用。

每个类都有构造方法,如果不显示的创建构造方法,编译器就会给这个生成一个没有参数的构造方法。

当父类里面没有写构造方法的时候,就被生成了没参数版本的构造方法。如果直接\dm{new}子类实例,就会调用刚才父类这个没参数版本的构造方法。

当父类里有构造方法的时候,并且这个构造方法带有参数的时候,编译器就不再自动生成无参数版本的构造方法,此时再创建子类实例,就需要显示的调用父类的构造方法,并且进行传参,否则创建不出父类的实例,就会编译出错。因此需要在子类的构造方法中显示的调用父类构造方法即可。(使用\dm{super}关键字)

从父类继承过来的属性,既可以用\dm{this.}的方式获取,也可以使用\dm{super.}的方式获取,获取到的是同一个属性。如果子类中新创建了一个相同的属性,则,只能通过\dm{super.}的方式获取父类的该属性,\dm{this.}获取到的是子类的对应属性。

子类继承父类后,子类需要先构造父类:创建子类实例的时候,先构造父类对象(执行父类构造方法的逻辑),再构造子类对象(执行子类构造方法的逻辑)。父类的构造方法必须放在第一行。

对象初始化的顺序:
\begin{itemize}
\itemsep=0pt \parskip =0pt
  \item 把父类的实例全部创建完再创建子类
  \item 就地初始化和代码块是并列关系,按照在代码中出现的顺序来进行初始化。
  \item 最后再执行构造方法中的代码
\end{itemize}

\subsubsection{\dm{protected}关键字}
\dm{protected}关键字修饰的变量可以被子类(无论是不是在同一个包内)访问,也可以被同一包下的其他类访问。

类前面加\dm{public}表示这个类可以被其他包使用,不加表示只能在当前包使用,不能加\dm{protected}和\dm{private}

\subsubsection{\dm{final}关键字}
\dm{final}关键字加在类前,可以显示的禁止类继承,以防止类继承被滥用。
\subsection{组合}
组合也是为了代码重用,也是面向对象的一个重要特性。一个类的成员也可以是其他的类。

将多个类的实例作为另一个类的字段。

组合表示\dm{has-a}语义
可以理解为一个学校\textbf{包含}若干老师、学生和教室等。实际开发中大部分场景使用组合。

继承表示\dm{is-a}语义,即\textbf{猫}是一种动物。
\subsection{多态}
\subsection{向上转型}
使用父类的引用指向一个子类的实例。向上转型可以省略强制类型转换。

不同类型的引用之间不能相互赋值,除非两者之间是父子关系。

向上转型发生的时机:
\begin{itemize}
\itemsep=0pt \parskip =0pt
\item 直接赋值
\item 方法传参
\item 方法返回
\end{itemize}
向上转型后,无法通过父类引用访问子类特有的属性和方法

\subsection{动态绑定}
如果父类中包含的方法在子类中有对应的同名同参数的方法,就会进行动态绑定。(这里静态指的是编译期,动态指的是运行时。)运行时决定要调用哪个方法。

如果方法只在父类存在,此时调用的方法就是父类的方法(不涉及动态绑定),如果方法只在子类中存在,此时调用方法就会编译报错(不涉及动态绑定)。

如果某个方法在父类和子类中都存在,并且参数相同,此时调用该方法就会涉及到动态绑定。在程序运行时,看当前引用究竟指向的是父类实例还是子类实例,指向父类实例就调用父类中的那个方法,指向子类就调用子类中的那个方法。

\begin{tips}
需要给子类的方法加注解\dm{@Override},显式的告诉编译器当前这个子类方法是重写了父类的方法。

注解是为了让编译器进行更好的检查和校验工作,加上注解\dm{@Override}就是明确的告诉编译器,我们的目的就是进行重写,防止无意中写出了这种方法重写的代码
\end{tips}

如果某个方法在父类和子类中都存在,但是参数不同,此时调用方法,不会涉及动态绑定,而是相当于方法重载。
\subsection{方法重写}
子类实现父类的同名方法,并且参数的类型和个数完全相同,这种情况称为覆写/重写/覆盖(Override)。
\begin{itemize}
\itemsep=0pt \parskip =0pt
\item 重写和重载不一样。
\item 普通方法可以重写,\dm{static}方法不能重写
\item 重写中子类的方法的访问权限不能低于父类方法的访问权限
\item  重写的方法返回值类型不一定和父类相同。(父类子类返回值要有一定的关系,假设返回值类型互不相干,就会编译出错,假设返回类型具有父子关系,就可以编译通过)
\end{itemize}
\subsection{理解多态}
多态是一种程序设计的思想方法,具体的语法体现,向上转型、方法重写、动态绑定。

多态:一个引用,对应到多种形态。多态的设计思想,本质上是“封装”的更进一步。
\subsection{向下转型}

把父类的引用转为子类的引用,向下转型必须保证操作合理,否则会存在问题。

使用场景:
\begin{itemize}
\itemsep=0pt \parskip =0pt
\item 有些方法只有子类有,父类没有,此时使用多态的方式则无法执行到对应的子类方法,就必须把父类的引用转回成子类的引用,然后调用对应的方法
\end{itemize}
使用\dm{instanceof}做出判断(\dm{if(父类引用 instanceof 子类引用)}),判定当前的父类的引用是不是指向该子类,如果不是就不进行向下转型。

\subsection{在构造方法中调用重写的一个方法}
1.A是B的父类,构造B的时候,就需要先构造A的实例

2.构造A的实例,就会调用A的构造方法

3.调用A的构造方法(该方法里调用了另外一个方法\dm{func()},\dm{func()}在B中重写)的时候,就会调用到\dm{this.func()},而此时的\dm{this}指向的是子类B的实例,触发了方法的动态绑定。

4.此时B中的初始化代码(包括就地初始化、代码块初始化以及构造方法的初始化都没有执行到)

因此,要避免在构造方法中调用其他可能被重写的方法。

\subsection{抽象类和接口}
抽象类和接口是搭配多态来使用的。

不需要实例化的类称为抽象类,本身没有方法体,只是为了被子类重写的方法,这种方法称为抽象方法。

给类前面加上\dm{abstract}关键字表示为抽象类,此时如果创建抽象类的实例则会编译报错。抽象类中可以有普通的属性和方法,可以有静态的属性和方法,可以继承其他的类,也可以被其他的类继承。

给方法前加上\dm{abstract}关键字表示为抽象方法,此时的方法不需要方法体,抽象方法只能在抽象类中存在(也可以在接口中存在),抽象方法的存在就是为了让子类进行重写。抽象方法不能是\dm{private}。

接口:抽象类的进一步,抽象类只是不能实例化,其他方面与普通类类似。接口不仅不能实例化,同时也不具备类的各种特性。接口的作用是为了解决Java不能多继承的问题。

\begin{itemize}
\itemsep=0pt \parskip =0pt
\item \dm{interface}关键字表示接口。\dm{public interface Shape{}}
\item 接口中可以放抽象方法(不用写\dm{abstract}关键字),接口中不能放普通的方法。
\item 接口中不能放普通的属性,只能放\dm{public static final}修饰的属性,同理,\dm{public static final}也可以省略。
\item 接口不能继承自其他的类,但是可以继承自其他的接口。
\item 接口不能被其他的类继承,只能被其他的类实现(使用\dm{implements}关键字)。
    \\ \dm{public class Circle implements Shape{ }}
\end{itemize}

抽象类和接口的对比:
\begin{itemize}
\itemsep=0pt \parskip =0pt
\item 抽象类和普通类差不多,只是不能实例化,接口与普通类相差很多(属性,方法和其他类的关系等)
\item 一个类只能继承自一个抽象类,但是一个类可以实现多个接口
\end{itemize}

接口命名一般使用I作为前缀,接口命名的时候,一般使用形容词词性的单词进行命名。

\begin{lstlisting}
//动物类
abstract pubilc Animal{
    protected String name;
    public Animal(String name){
        this.name = name;
    }
}
//跑的接口
public interface IRunning{
    abstract void run();
}
//飞的接口
public interface IFlying {
    void fly();
}

//小鸟类(继承动物的类并实现跑的接口)
public class Bird extends Animal implements IRunning,IFlying{ // 多个接口以逗号隔开
    public Bird(String name) {
        super(name);
    }

    @Override
    public void run() {
        System.out.println(this.name + "一跳一跳的跑");
    }

    @Override
    public void fly() {
        System.out.println(this.name + "飞起来了");
    }
}

\end{lstlisting}
\begin{tips}
接口相当于是一种约束,要求了实现该接口的类,必须重写所有接口中的抽象方法。
\end{tips}

接口和接口之间是可以继承的。(说是继承,表示成组合更合适点)
\begin{lstlisting}
//跑的接口
public interface IRunning{
    abstract void run();
}
//游泳的接口
public interface ISwimming {
    void swim();
}

//两栖类接口
public interface IAmphibious extends IRunning ,ISwimming{
    //此时该接口就同时包含了IRunning中的抽象方法
    //也同时包含了ISwimming中的抽象方法
}
\end{lstlisting}
\section{处理异常}
异常是程序\textbf{运行}过程中出现的一种错误。

防御式编程:
\begin{itemize}
\itemsep=0pt \parskip =0pt
\item LBYL(Look Before You Leap):操作之前做充分的检查,检查完上一步之后,再来做下一步的操作。如果上一步失败,就不继续执行
\item EAFP(It's Easier Ask Forgiveness than Permission):事后获取原谅比事前获取许可更简单,先斩后奏。
\end{itemize}

\begin{lstlisting}
//EAFP风格的语法

try{
    //有可能出现异常的语句
}[catch (异常类型 异常对象){
}......]
[finally {
    //异常出口
}]
\end{lstlisting}
异常的具体语法:
\begin{itemize}
\itemsep=0pt \parskip =0pt
\item \dm{try}关键字:\dm{try}语句块中放置可能会抛出异常的代码。
\item \dm{catch}语句块中放置用来处理异常的代码,当\dm{try}中出现异常的时候,就去会进入\dm{catch}中执行
\item \dm{throw}:主动抛出一个Java异常(Java的异常本质上就是一个一个的对象)。
\item \dm{throws}:某个方法可能会抛出某些异常。
\item \dm{finally}:一般用于异常处理完毕后的收尾工作。
\end{itemize}
\begin{lstlisting}
try{
    System.out.println("try 中异常之前的代码");
    int[] a = null;
    System.out.println(a[0]);
    System.out.println("try中异常之后的代码");
}catch (NullPointerException e){ //e为形参
    System.out.println("catch中的代码");
    System.out.println("e中的信息");
    //这个方法能够打印出当前出现异常的代码对应的调用栈的信息
    e.printStackTrace();
}
\end{lstlisting}
上边代码中e类似于一个形参,当\dm{try}的代码抛出一个异常之后,e就对应着这个异常。通过e就可以获取到异常的一些具体信息(哪个代码中出现了异常)

如果\dm{try}中可能抛出多种异常的化,也就需要多个\dm{catch}语句来进行处理,多个\dm{catch}语句与多分支语句类似。
\begin{lstlisting}
try{
    //有可能出现异常的语句
}catch (异常类型1 异常对象1){
    //处理异常代码1
}catch (异常类型2 异常对象2){
    //处理异常代码2
}
\end{lstlisting}

使用一个\dm{catch}语句捕获多个异常:如果程序对于多个异常的处理逻辑是一样的,就可以使用此方式。抛出这若干个异常中的任何一个,都会触发\dm{catch}
\begin{lstlisting}
try{
    //有可能出现异常的语句
}catch (异常类型1 | 异常类型2 异常对象1){
    //处理异常代码1
}
\end{lstlisting}

\dm{finally}中的逻辑无论是前面的代码中是否触发异常,都会执行到

\dm{throw}:20视频的20分钟-24分钟

\dm{throws}:标注当前的方法,可能抛出什么样的异常
\begin{lstlisting}
public static int divide(int x, int y) throws ArithmeticException{
    if(y == 0){
        throw new ArithmeticException("抛出0异常");
    }
    return x / y;
}
\end{lstlisting}
\section{泛型}
\begin{lstlisting}
class 泛型类名称<类型形参列表>{
    //可以使用的参数列表
}

class ClassName<T1, T2, T3, ...,Tn>{
    //类体
}
\end{lstlisting}
类型形参常用的名称
\begin{itemize}
\itemsep=0pt \parskip =0pt
\item E表示Element
\item K表示Key
\item V表示Value
\item N表示Number
\item T表示Type
\item S,U,V等等:第二、第三、第四个类型
\end{itemize}

实例:

\begin{lstlisting}
package Wsn;

public class Myarray<E> {
    private E[] array = null;
    private int size;
    private int capacity;

    public Myarray(int capacity) {
        //由于E的类型不确定,无法创建E的实例,所以要创建Object类型然后强制转换成E类型
        array = (E[]) new Object[capacity];
        size = 0;
        this.capacity = capacity;
    }
    public void add(E data){
        if (size < capacity) {
            array[size++] = data;
        }
    }
    public E get(int index){
        return array[index];
    }
    public int size(){
        return size;
    }

    public static void main(String[] args){
        Myarray<String> str = new Myarray<>(10);
        str.add("hello");
        str.add("word");
        int size = str.size();
        String str1 = str.get(0);
        String str2 = str.get(1);
        System.out.println("当前数组元素的个数是:" + size);
        System.out.println("它们分别是:"+ "[" + str1 + "," + str2 + "]");
    }
}
\end{lstlisting}
\begin{itemize}
\itemsep=0pt \parskip =0pt
\item 使用泛型后,针对对象实例化的时候就需要填写泛型参数的实际类型
\item Java的泛型只能是引用类型,如果是内置类型,必须使用其对应的包装类(包装类也是引用类型)
\end{itemize}
\subsection{类型边界}
定义泛型类的时候,对未来实例化的时候能传入的实参类型作出限制

\begin{lstlisting}
class 泛型类名称<E extends U>{
    //类体
}
\end{lstlisting}

实例化时,E只能是U或U的子类,否则会报错。
\newpage
%T(n) < \sqrt{5n\cdot [3n\cdot (n+n)+4]+6} = \sqrt{5n\cdot (6n^2+4)+6}<\sqrt{5n\cdot (6n^2+n^2)+6}=

%\sqrt{35n^3+6}<\sqrt{35n^3+n^3}=\sqrt{36n^3}=6n^{1.5}=O(n^{1.5})
%T_{2}(n)=1^{2}+2^{2}+\cdots+n^{2}=\dfrac{n(n+1)(2n+1)}{6}=O(n^3)
%\\
%& T_{3}(n)=1^{3}+2^{3}+\cdots+n^{3}=\dfrac{n^2(n+1)^2}{4}=O(n^4)
%\\
%& T_{4}(n)=1^{4}+2^{4}+\cdots+n^{4}=\dfrac{n(n+1)(2n+1)(3n^2+3n-1)}{30}=O(n^5)

%&T_{a}(n)=a^{0}+a^{1}+\cdots+a^{n}=\frac{a^{n+1}-1}{a-1} =O(a^{n})\\
%&1+2+4+\cdots+2^{n}=2^{n+1}-1=O(2^{n+1})=O(2^n)
%\log 1+\log 2+\log 3+\cdots+\log n=\log(n!)=O(nlog n)
%&T(n) = T(n-1)+O(1)\\
%&T(0)= O(1)
%T(n)-n=T(n-1)-(n-1)\makebox[0pt][l]{\text{等式两边同时减$n$,因为$O(1)$是常数次操作,把其当做常数1}}
%T(n)-n&=T(n-1)-(n-1)\text{等式两边同时减$n$,因为$O(1)$是常数次操作,把其当做常数1}\\
%&=T(n-2)-(n-2)=\cdots\\
%&=T(2)-2\\\approx \div
%&=T(1)-1=T(0) - 0 =O(1)
%T(n)= n+O(1)=O(n)
%T(0) = T(1) =1 = O(1)

%fib(n) = \frac{1}{\sqrt{5}}\left[\left(\frac{1+\sqrt{5}}{2}\right)^{n}-\left(\frac{1-\sqrt{5}}{2}\right)^{n}\right]


\begin{align*}
%S(0) = \frac{T(0)+1}{2} = 1 = fib(1)~~~~S(1) = 1 = fib(2)
%S(n) &= \frac{\textcolor[rgb]{1.00,0.00,0.00}{T(n)}+1}{2}\\
%&=\frac{\textcolor[rgb]{1.00,0.00,0.00}{T(n-1)+T(n-2)+1}+1}{2}\\
%&=\frac{T(n-1)+1}{2}+\frac{T(n-2)+1}{2}=S(n-1)+S(n-2)
&S(n) = fib(n+1) = \frac{1}{\sqrt{5}}\left[\left(\frac{1+\sqrt{5}}{2}\right)^{n+1}-\left(\frac{1-\sqrt{5}}{2}\right)^{n+1}\right]\\
&T(n)=2S(n) - 1=\frac{2}{\sqrt{5}}\left[\left(\frac{1+\sqrt{5}}{2}\right)^{n+1}-\left(\frac{1-\sqrt{5}}{2}\right)^{n+1}\right]-1<\frac{4}{\sqrt{5}}\left(\frac{1+\sqrt{5}}{2}\right)^{n+1}<2\times1.62^{n+1}=O(1.62^n)
\end{align*}
\end{document} 