\documentclass[a4paper]{report}
\usepackage[space,fancyhdr,fntef]{ctexcap}
\usepackage{fontspec}
\fontspec{宋体}
\setmainfont{Times New Roman}
%\fontsize{50pt}{50pt}\selectfont
\renewcommand{\rmdefault}{ptm}
\usepackage[namelimits,sumlimits,nointlimits]{amsmath}
\usepackage[text={169mm,250mm},bottom=20mm,top=25mm,left=25mm,right=15mm,centering]{geometry}
\usepackage{color}
\usepackage{CJKfntef}%下划线宏包160页
\usepackage{xcolor}
\usepackage{arydshln}%234页,虚线表格宏包
\pagestyle{fancy} \fancyhf{}
\fancyhead[OC]{\color{gray}\rightmark}

\fancyhead[EC]{\color{gray}\leftmark}
\fancyfoot[C]{\color{gray}\thepage}
\renewcommand{\headrule}{\color{gray}\hrule width\headwidth}
%\renewcommand{\footrulewidth}{0.4pt}%改为0pt即可去掉页脚上面的横线
%\usepackage{parskip}
%\usepackage{indentfirst}
\usepackage{graphicx}%插图宏包,参见手册318页
\definecolor{dkgreen}{RGB}{106,135,89}
\definecolor{dkblue}{RGB}{103,150,163}
\definecolor{wgray}{RGB}{248,248,248}
\definecolor{WGRAY}{RGB}{248,248,248}
\usepackage{listings}
\lstset{language=Java,
backgroundcolor=\color{wgray},
rulesepcolor=\color{red!20!green!20!blue!20},%代码块边框为淡青色
%lablestep=1,
%lablesep=5pt,
%lablestyle=\tiny,
%tablesize=4,
%captionpos=b,
basicstyle=\ttfamily\small,
keywordstyle=\color{orange},
commentstyle=\color{gray},
stringstyle=\color{dkgreen},
numberstyle=\tiny,
numbersep=8pt,
frame=single,%topline.bottomline,lines,single,leftline
identifierstyle=\color{dkblue},
numbers=left,
stepnumber=1,
xleftmargin=2em,xrightmargin=2em, aboveskip=1em,
breaklines=true
}
\usepackage[xetex,colorlinks]{hyperref}%394页  \href{网址}{文本}
\hypersetup{urlcolor=blue}
%\linebreak[2]%换行,152页
\usepackage{fancybox}%盒子宏包55页
\setcounter{secnumdepth}{4}
\CTEXoptions[contentsname={目\hspace{15pt}录}]
\CTEXsetup[beforeskip={-40pt},afterskip={20pt plus 2pt minus 2pt}]{chapter}

%目录设置
\usepackage{titletoc}
\usepackage[toc]{multitoc}
\titlecontents{chapter}[4em]{\addvspace{2.3mm}\bf}{\contentslabel{4.0em}}{}{\titlerule*[5pt]{$\cdot$}\contentspage}
\titlecontents{section}[4em]{}{\contentslabel{2.5em}}{}{\titlerule*[5pt]{$\cdot$}\contentspage}
\titlecontents{subsection}[7.2em]{}{\contentslabel{3.3em}}{}{\titlerule*[5pt]{$\cdot$}\contentspage}
\usepackage{fancyvrb}%75页抄录宏包
\begin{document}
\flushbottom%版心底部对齐
\newcommand{\dm}[1]{\colorbox{wgray}{\lstinline`#1`}}
\newcommand{\myroman}[1]{\uppercase\expandafter{\romannumeral#1}}
\newcounter{num}[section] \renewcommand{\thenum}{\arabic{num}.} \newcommand{\num}{\refstepcounter{num}\text{\thenum}}

\newenvironment{tips}{\kaishu\zihao{-6}\color{blue}{\noindent\rule[-3pt]{\textwidth}{0.5pt}\par \em \noindent {\zihao{-5} \textcolor[rgb]{1.00,0.00,0.00}{Tips}}}\par}{\\ \rule[1mm]{\textwidth}{0.5pt}\par}

\newenvironment{zhengming}{\kaishu\zihao{-5}\color{blue}{\noindent\em 证明:}\par}{\hfill $\diamondsuit$\par}

\tableofcontents
\pagenumbering{Roman}%设置目录页码
\clearpage
\pagenumbering{arabic}%设置正文页码

\num \href{https://leetcode-cn.com/problems/reverse-linked-list/}{反转一个单链表}



\begin{lstlisting}[title = 示例]
输入: 1->2->3->4->5->NULL
输出: 5->4->3->2->1->NULL
\end{lstlisting}

\begin{tips}
1.链表为空(\dm{head == null})或者链表只有一个元素(\dm{head.next == null})。反转后还是自身:\dm{return head;}

2.链表反转:首先用一个引用储存当前元素的下一个元素。\dm{nextNode = curNode;}然后把当前元素\dm{curNode}的\dm{next}指向它的前一个元素\\ \dm{prevNode}:\dm{curNode.next = prevNode}

3.第2步执行完后,处理下一个元素:\dm{prevNode = curNode;curNode = curNode.next}。

4.\dm{curNode == null}时,说明到了链表的末端(循环退出条件)。此时\dm{prevNode}指向的是原链表的最后一个元素,即新链表的首节点:\\ \dm{return prevNode;}

5.初始化:第一个元素没有前驱结点,反转后,原链表第一个节点为新链表的最后一个节点,新链表的最后一个节点的\dm{next}应指向\dm{null}。所以\dm{prevNode}初始化为\dm{null}:\dm{Node prevNode = null;},\dm{curNode}最开始应指向链表的首节点:\dm{Node curNode = head;}。
\end{tips}

\begin{lstlisting}[title = 迭代版本]
/**
 * Definition for singly-linked list.
 * public class ListNode {
 *     int val;
 *     ListNode next;
 *     ListNode() {}
 *     ListNode(int val) { this.val = val; }
 *     ListNode(int val, ListNode next) { this.val = val; this.next = next; }
 * }
 */
class Solution {
    public ListNode reverseList(ListNode head) {
        if(head == null || head.next == null){
            return head;
        }
        ListNode prevNode = null;
        ListNode curNode = head;

        while(curNode != null){
            ListNode nextNode = curNode.next;
            curNode.next = prevNode;
            prevNode = curNode;
            curNode = nextNode;
        }
        return prevNode;
    }
}
\end{lstlisting}

\begin{tips}
1.可以将问题分为一个头结点和链表的剩余部分。

2.头结点为问题的平凡解:链表只有一个节点,直接返回。

3.子问题的解:头节点(\dm{head})的下一个节点\dm{head.next}的\dm{next}指向头结点。

4.头结点指向\dm{null}
\end{tips}

\begin{lstlisting}[title = 递归版本]
class Solution {
    public ListNode reverseList(ListNode head) {
        if(head == null || head.next == null){
            return head;
        }
        ListNode p = reverseList(head.next);
        head.next.next = head;
        head.next = null;
        return p;
    }
}
\end{lstlisting}

\num \href{https://leetcode-cn.com/problems/remove-linked-list-elements/}{移除链表元素}

删除链表中等于给定值 val 的所有节点。
\begin{lstlisting}[title = 示例]
输入: 1->2->6->3->4->5->6, val = 6
输出: 1->2->3->4->5
\end{lstlisting}

\begin{tips}
要考虑链表为空的情况。链表为空,直接返回\dm{null}

使用傀儡节点\dm{dummy}来进行辅助删除。\dm{dummy.next = head}

两个指针,一个指向待删除元素(\dm{cur})。一个指向待删除元素的前一个位置(\dm{prev})。

如果找到了待删除元素:将该元素前一个位置(\dm{pre})的\dm{next}指向删除元素的\dm{next}:\dm{pre.next = cur.next}。然后将\dm{cur}复位,即指向\dm{pre.next}:\dm{cur = pre.next}

如果没有找到,两个指针同时向后移动:\dm{pre = pre.next;cur = cur.next;}

循环结束的条件:\dm{cur}指向空,表示已经找到了链表的最后。

最后返回\dm{dummy.next}
\end{tips}

\begin{lstlisting}
/**
 * Definition for singly-linked list.
 * public class ListNode {
 *     int val;
 *     ListNode next;
 *     ListNode() {}
 *     ListNode(int val) { this.val = val; }
 *     ListNode(int val, ListNode next) { this.val = val; this.next = next; }
 * }
 */
class Solution {
    public ListNode removeElements(ListNode head, int val) {
        if (head == null){
            return null;
        }
        ListNode dummy = new ListNode(0,head);
        ListNode prev = dummy;
        ListNode cur = dummy.next;
        while(cur != null){
            if(cur.val == val){
                prev.next = cur.next;
                cur = prev.next;
            }else{
                prev = prev.next;
                cur = cur.next;
            }
        }
        return dummy.next;
    }
}
\end{lstlisting}

\num \href{https://leetcode-cn.com/problems/pascals-triangle/}{杨辉三角}

给定一个非负整数\dm{numRows},生成杨辉三角的前\dm{numRows}行。

\begin{lstlisting}
class Solution {
    public List<List<Integer>> generate(int numRows) {

        List<List<Integer>> pasTri = new ArrayList<>();
        if(numRows == 0){
            return pasTri;
        }
        List<Integer> first = new ArrayList<>();
        first.add(1);
        pasTri.add(first);
        if(numRows == 1){
            return pasTri;
        }
        List<Integer> second = new ArrayList<>();
        second.add(1);
        second.add(1);
        pasTri.add(second);
        if(numRows == 2){
            return pasTri;
        }
        for(int i = 3; i <= numRows; i++){
            List<Integer> list1 = new ArrayList<>();
            list1.add(1);
            for(int j = 2; j <= i - 1; j++){
                int a = pasTri.get(i - 1 - 1).get(j - 1);
                int b = pasTri.get(i - 1 - 1).get(j - 1 - 1);
                list1.add(a + b);
            }
            list1.add(1);
            pasTri.add(list1);
        }
        return pasTri;
    }
}
\end{lstlisting}

\num \href{https://leetcode-cn.com/problems/three-consecutive-odds/}{存在连续三个奇数的数组}

给你一个整数数组\dm{arr},请你判断数组中是否存在连续三个元素都是奇数的情况:如果存在,请返回\dm{true};否则,返回\dm{false}。
\begin{lstlisting}[title=示例1 ]
输入:arr = [2,6,4,1]
输出:false
解释:不存在连续三个元素都是奇数的情况。
\end{lstlisting}
\begin{lstlisting}[title=示例2]
输入:arr = [1,2,34,3,4,5,7,23,12]
输出:true
解释:存在连续三个元素都是奇数的情况,即 [5,7,23] 。
\end{lstlisting}

\begin{lstlisting}[title=版本1:LeetCode提供的]
class Solution {
    public boolean threeConsecutiveOdds(int[] arr) {
        for(int i = 0; i < arr.length - 2; i++){
            if((arr[i] % 2 == 1) && (arr[i+1] % 2 == 1) && (arr[i + 2] % 2 == 1)){
                return true;
            }
        }
        return false;
    }
}
\end{lstlisting}

\begin{lstlisting}[title=版本2:自己使用顺序表实现]
public class MyArrayList1 {
    public boolean threeConsecutiveOdds(List<Integer> arr) {
        for (int i = 0; i < arr.size() - 2; i++) {
            if ((arr.get(i) % 2 == 1) && (arr.get(i + 1) % 2 == 1) && (arr.get(i + 2) % 2 == 1)){
                return true;
            }
        }
        return false;
    }
}
\end{lstlisting}
\num \href{https://leetcode-cn.com/problems/employee-importance/}{员工的重要性}

给定一个保存员工信息的数据结构,它包含了员工唯一的id,重要度和直系下属的id。
比如,员工1是员工2的领导,员工2是员工3的领导。他们相应的重要度为15, 10, 5。那么员工1的数据结构是[1, 15, [2]],员工2的数据结构是[2, 10, [3]],员工3的数据结构是[3, 5, []]。注意虽然员工3也是员工1的一个下属,但是由于并不是直系下属,因此没有体现在员工1的数据结构中。

现在输入一个公司的所有员工信息,以及单个员工id,返回这个员工和他所有下属的重要度之和。

\begin{lstlisting}
示例 1:

输入: [[1, 5, [2, 3]], [2, 3, []], [3, 3, []]], 1
输出: 11
解释:
员工1自身的重要度是5,他有两个直系下属2和3,而且2和3的重要度均为3。因此员工1的总重要度是 5 + 3 + 3 = 11。
注意:
\end{lstlisting}
一个员工最多有一个直系领导,但是可以有多个直系下属,员工数量不超过2000。

\begin{tips}
1.遍历整个员工列表employees,找到符合id的员工employee

2.这个员工employee如果没有下属employee.subordinates.size()==0,重要度就是自己的重要度employee.importance。

3.如果这个员工有下属,算出每个下属及下属的重要度。
\end{tips}

\begin{lstlisting}
/*
// Definition for Employee.
class Employee {
    public int id;
    public int importance;
    public List<Integer> subordinates;
};
*/

class Solution {
    public int getImportance(List<Employee> employees, int id) {

        for(int i = 0;i < employees.size();i++){
            Employee employee = employees.get(i);
            if(employee.id == id){
                if(employee.subordinates.size() == 0){//没有下属
                    return employee.importance;
                }
                for(int j = 0; j < employee.subordinates.size(); j++){
                    employee.importance += getImportance(employees,employee.subordinates.get(j));
                }
                return employee.importance;
            }
        }
        return 0;
    }
}
\end{lstlisting}

\end{document} 