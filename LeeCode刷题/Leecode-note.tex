\documentclass[a4paper]{report}
\usepackage[space,fancyhdr,fntef]{ctexcap}
\usepackage{fontspec}
\fontspec{宋体}
\setmainfont{Times New Roman}
%\fontsize{50pt}{50pt}\selectfont
\renewcommand{\rmdefault}{ptm}
\usepackage[namelimits,sumlimits,nointlimits]{amsmath}
\usepackage[text={169mm,250mm},bottom=20mm,top=25mm,left=25mm,right=15mm,centering]{geometry}
\usepackage{color}
\usepackage{CJKfntef}%下划线宏包160页
\usepackage{xcolor}
\usepackage{arydshln}%234页,虚线表格宏包
\pagestyle{fancy} \fancyhf{}
\fancyhead[OC]{\color{gray}\rightmark}

\fancyhead[EC]{\color{gray}\leftmark}
\fancyfoot[C]{\color{gray}\thepage}
\renewcommand{\headrule}{\color{gray}\hrule width\headwidth}
%\renewcommand{\footrulewidth}{0.4pt}%改为0pt即可去掉页脚上面的横线
%\usepackage{parskip}
%\usepackage{indentfirst}
\usepackage{graphicx}%插图宏包,参见手册318页
\definecolor{dkgreen}{RGB}{106,135,89}
\definecolor{dkblue}{RGB}{103,150,163}
\definecolor{wgray}{RGB}{248,248,248}
\definecolor{WGRAY}{RGB}{248,248,248}
\usepackage{listings}
\lstset{language=Java,
backgroundcolor=\color{wgray},
rulesepcolor=\color{red!20!green!20!blue!20},%代码块边框为淡青色
%lablestep=1,
%lablesep=5pt,
%lablestyle=\tiny,
%tablesize=4,
%captionpos=b,
basicstyle=\ttfamily\small,
keywordstyle=\color{orange},
commentstyle=\color{gray},
stringstyle=\color{dkgreen},
numberstyle=\tiny,
numbersep=8pt,
frame=single,%topline.bottomline,lines,single,leftline
identifierstyle=\color{dkblue},
numbers=left,
stepnumber=1,
xleftmargin=2em,xrightmargin=2em, aboveskip=1em,
breaklines=true
}
\usepackage[xetex,colorlinks]{hyperref}%394页  \href{网址}{文本}
\hypersetup{urlcolor=blue}
%\linebreak[2]%换行,152页
\usepackage{fancybox}%盒子宏包55页
\setcounter{secnumdepth}{4}
\CTEXoptions[contentsname={目\hspace{15pt}录}]
\CTEXsetup[beforeskip={-40pt},afterskip={20pt plus 2pt minus 2pt}]{chapter}

%目录设置
\usepackage{titletoc}
\usepackage[toc]{multitoc}
\titlecontents{chapter}[4em]{\addvspace{2.3mm}\bf}{\contentslabel{4.0em}}{}{\titlerule*[5pt]{$\cdot$}\contentspage}
\titlecontents{section}[4em]{}{\contentslabel{2.5em}}{}{\titlerule*[5pt]{$\cdot$}\contentspage}
\titlecontents{subsection}[7.2em]{}{\contentslabel{3.3em}}{}{\titlerule*[5pt]{$\cdot$}\contentspage}
\usepackage{fancyvrb}%75页抄录宏包
\begin{document}
\flushbottom%版心底部对齐
\newcommand{\dm}[1]{\colorbox{wgray}{\lstinline`#1`}}
\newcommand{\myroman}[1]{\uppercase\expandafter{\romannumeral#1}}
\newcounter{num}[section] \renewcommand{\thenum}{\arabic{num}.} \newcommand{\num}{\refstepcounter{num}\text{\thenum}}

\newenvironment{tips}{\kaishu\zihao{-6}\color{blue}{\noindent\rule[-3pt]{\textwidth}{0.5pt}\par \em \noindent {\zihao{-5} \textcolor[rgb]{1.00,0.00,0.00}{Tips}}}\par}{\\ \rule[1mm]{\textwidth}{0.5pt}\par}

\newenvironment{zhengming}{\kaishu\zihao{-5}\color{blue}{\noindent\em 证明:}\par}{\hfill $\diamondsuit$\par}

\tableofcontents
\pagenumbering{Roman}%设置目录页码
\clearpage
\pagenumbering{arabic}%设置正文页码
\num 员工的重要性\href{https://leetcode-cn.com/problems/employee-importance/}{员工的重要性}

给定一个保存员工信息的数据结构,它包含了员工唯一的id,重要度和直系下属的id。
比如,员工1是员工2的领导,员工2是员工3的领导。他们相应的重要度为15, 10, 5。那么员工1的数据结构是[1, 15, [2]],员工2的数据结构是[2, 10, [3]],员工3的数据结构是[3, 5, []]。注意虽然员工3也是员工1的一个下属,但是由于并不是直系下属,因此没有体现在员工1的数据结构中。

现在输入一个公司的所有员工信息,以及单个员工id,返回这个员工和他所有下属的重要度之和。

\begin{lstlisting}
示例 1:

输入: [[1, 5, [2, 3]], [2, 3, []], [3, 3, []]], 1
输出: 11
解释:
员工1自身的重要度是5,他有两个直系下属2和3,而且2和3的重要度均为3。因此员工1的总重要度是 5 + 3 + 3 = 11。
注意:
\end{lstlisting}
一个员工最多有一个直系领导,但是可以有多个直系下属,员工数量不超过2000。

\begin{tips}
1.遍历整个员工列表employees,找到符合id的员工employee

2.这个员工employee如果没有下属employee.subordinates.size()==0,重要度就是自己的重要度employee.importance。

3.如果这个员工有下属,算出每个下属及下属的重要度。
\end{tips}

\begin{lstlisting}
/*
// Definition for Employee.
class Employee {
    public int id;
    public int importance;
    public List<Integer> subordinates;
};
*/

class Solution {
    public int getImportance(List<Employee> employees, int id) {

        for(int i = 0;i < employees.size();i++){
            Employee employee = employees.get(i);
            if(employee.id == id){
                if(employee.subordinates.size() == 0){//没有下属
                    return employee.importance;
                }
                for(int j = 0; j < employee.subordinates.size(); j++){
                    employee.importance += getImportance(employees,employee.subordinates.get(j));
                }
                return employee.importance;
            }
        }
        return 0;
    }
}
\end{lstlisting}

\end{document} 