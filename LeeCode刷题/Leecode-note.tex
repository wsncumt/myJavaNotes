\documentclass[a4paper]{report}
\usepackage[space,fancyhdr,fntef]{ctexcap}
\usepackage{array}
\usepackage{fontspec}
\fontspec{宋体}
\setmainfont{Times New Roman}
%\fontsize{50pt}{50pt}\selectfont
\renewcommand{\rmdefault}{ptm}
\usepackage[namelimits,sumlimits,nointlimits]{amsmath}
\usepackage[text={169mm,250mm},bottom=20mm,top=25mm,left=25mm,right=15mm,centering]{geometry}
\usepackage{color}
\usepackage{CJKfntef}%下划线宏包160页
\usepackage{xcolor}
\usepackage{arydshln}%234页,虚线表格宏包
\pagestyle{fancy} \fancyhf{}
\fancyhead[OC]{\color{gray}\rightmark}

\fancyhead[EC]{\color{gray}\leftmark}
\fancyfoot[C]{\color{gray}\thepage}
\renewcommand{\headrule}{\color{gray}\hrule width\headwidth}
%\renewcommand{\footrulewidth}{0.4pt}%改为0pt即可去掉页脚上面的横线
%\usepackage{parskip}
%\usepackage{indentfirst}
\usepackage{graphicx}%插图宏包,参见手册318页
\definecolor{dkgreen}{RGB}{106,135,89}
\definecolor{dkblue}{RGB}{103,150,163}
\definecolor{wgray}{RGB}{248,248,248}
\definecolor{WGRAY}{RGB}{248,248,248}
\usepackage{listings}
\lstset{language=Java,
backgroundcolor=\color{wgray},
rulesepcolor=\color{red!20!green!20!blue!20},%代码块边框为淡青色
%lablestep=1,
%lablesep=5pt,
%lablestyle=\tiny,
%tablesize=4,
%captionpos=b,
basicstyle=\ttfamily\small,
keywordstyle=\color{orange},
commentstyle=\color{gray},
stringstyle=\color{dkgreen},
numberstyle=\tiny,
numbersep=8pt,
frame=single,%topline.bottomline,lines,single,leftline
identifierstyle=\color{dkblue},
numbers=left,
stepnumber=1,
xleftmargin=2em,xrightmargin=2em, aboveskip=1em,
breaklines=true
}
\usepackage[xetex,colorlinks]{hyperref}%394页  \href{网址}{文本}
\hypersetup{urlcolor=blue}
%\linebreak[2]%换行,152页
\usepackage{fancybox}%盒子宏包55页
\setcounter{secnumdepth}{4}
\CTEXoptions[contentsname={目\hspace{15pt}录}]
\CTEXsetup[beforeskip={-40pt},afterskip={20pt plus 2pt minus 2pt}]{chapter}

%目录设置
\usepackage{titletoc}
\usepackage[toc]{multitoc}
\titlecontents{chapter}[4em]{\addvspace{2.3mm}\bf}{\contentslabel{4.0em}}{}{\titlerule*[5pt]{$\cdot$}\contentspage}
\titlecontents{section}[4em]{}{\contentslabel{2.5em}}{}{\titlerule*[5pt]{$\cdot$}\contentspage}
\titlecontents{subsection}[7.2em]{}{\contentslabel{3.3em}}{}{\titlerule*[5pt]{$\cdot$}\contentspage}
\usepackage{fancyvrb}%75页抄录宏包
\begin{document}
\flushbottom%版心底部对齐
\newcommand{\dm}[1]{\colorbox{wgray}{\lstinline`#1`}}
\newcommand{\myroman}[1]{\uppercase\expandafter{\romannumeral#1}}
\newcounter{num}[section] \renewcommand{\thenum}{\arabic{num}.} \newcommand{\num}{\refstepcounter{num}\text{\thenum}}

\newenvironment{tips}{\kaishu\zihao{-6}\color{blue}{\noindent\rule[-3pt]{\textwidth}{0.5pt}\par \em \noindent {\zihao{-5} \textcolor[rgb]{1.00,0.00,0.00}{Tips}}}\par}{\\ \rule[1mm]{\textwidth}{0.5pt}\par}

\newenvironment{zhengming}{\kaishu\zihao{-5}\color{blue}{\noindent\em 证明:}\par}{\hfill $\diamondsuit$\par}

\tableofcontents
\pagenumbering{Roman}%设置目录页码
\clearpage
\pagenumbering{arabic}%设置正文页码
\num 给定一个按照升序排列的整数数组\dm{nums},和一个目标值 \dm{target}。找出给定目标值在数组中的开始位置和结束位置。
如果数组中不存在目标值\dm{target},返回 $[-1, -1]$。
\href{https://leetcode-cn.com/problems/find-first-and-last-position-of-element-in-sorted-array/} {OJ链接-在排序数组中查找元素的第一个和最后一个位置}。

\begin{tips}
使用二分查找。

关键在于等于时的情况处理:

寻找下界:如果\dm{mid}处的值等于目标值,判断\dm{mid}的左侧值是否等于目标值,如果左侧没有值或者左侧不等于目标值,说明该\dm{mid}为下界,否则需要从\dm{[lo,mid - 1]}区间来查找下界,即\dm{hi = mid - 1}

寻找上界:与寻找下界逻辑相反,如果\dm{mid}处的值等于目标值,判断\dm{mid}的右侧值是否等于目标值,如果右侧没有值或者右侧不等于目标值,说明该\dm{mid}为上界,否则需要从\dm{[mid + 1, hi]}区间来查找上界,即\dm{lo = mid + 1}
\end{tips}

\begin{lstlisting}[title=在排序数组中查找元素的第一个和最后一个位置]
class Solution {
    private int findFirst(int[] nums, int target) {
        int lo = 0;
        int hi = nums.length - 1;
        int mid = 0;
        while(lo <= hi){
            mid = lo + ((hi - lo) >> 1);
            if(nums[mid] < target){
                lo = mid + 1;
            }else if(target < nums[mid]){
                hi = mid - 1;
            }else{
                if(mid == 0 || nums[mid - 1] != target){
                    return mid;
                }else{
                    hi = mid - 1;
                }
            }
        }
        return -1;
    }

    private int findLast(int[] nums, int target) {
        int lo = 0;
        int hi = nums.length - 1;
        int mid = 0;
        while(lo <= hi){
            mid = lo + ((hi - lo) >> 1);
            if(nums[mid] < target){
                lo = mid + 1;
            }else if(target < nums[mid]){
                hi = mid - 1;
            }else{
                if(mid == nums.length - 1 || nums[mid + 1] != target){
                    return mid;
                }else{
                    lo = mid + 1;
                }
            }
        }
        return -1;
    }
    public int[] searchRange(int[] nums, int target) {
        return new int[]{findFirst(nums,target),findLast(nums,target)};

    }
}
\end{lstlisting}

\num 根据\textbf{逆波兰表示法},求表达式的值。
有效的运算符包括\dm{+},\dm{-},\dm{*},\dm{/} 。每个运算对象可以是整数,也可以是另一个逆波兰表达式。\href{https://leetcode-cn.com/problems/evaluate-reverse-polish-notation/} {OJ链接-逆波兰表达式求值}

\textbf{说明}:
\begin{itemize}
\itemsep=0pt \parskip=0pt
  \item 整数除法只保留整数部分。
  \item 给定逆波兰表达式总是有效的。换句话说,表达式总会得出有效数值且不存在除数为 0 的情况。
\end{itemize}

\begin{tips}
使用题目给的提示:适合用栈操作运算:遇到数字则入栈;遇到算符则取出栈顶两个数字进行计算,并将结果压入栈中。
\end{tips}

\begin{lstlisting}[title=逆波兰表达式求值]
class Solution {
    private boolean isNumber(String str){
        return !"+".equals(str) && !"-".equals(str) && !"*".equals(str) && !"/".equals(str);
    }

    public int evalRPN(String[] tokens) {
        Deque<Integer> stk = new ArrayDeque<>();
        for(String str : tokens){
            if(isNumber(str)){
                stk.push(Integer.valueOf(str));
                continue;
            }
            int n1 = stk.pop();
            int n2 = stk.pop();
            switch(str){
                case "+":
                    stk.push(n2 + n1);
                    break;
                case "-":
                    stk.push(n2 - n1);
                    break;
                case "*":
                    stk.push(n2 * n1);
                    break;
                case "/":
                    stk.push(n2 / n1);
                    break;
            }

        }
        return stk.pop();
    }
}
\end{lstlisting}

\num 根据\textbf{逆波兰表示法},求表达式的值。
有效的运算符包括\dm{+},\dm{-},\dm{*},\dm{/} 。每个运算对象可以是整数,也可以是另一个逆波兰表达式。\href{https://leetcode-cn.com/problems/evaluate-reverse-polish-notation/} {OJ链接-逆波兰表达式求值}

\textbf{说明}:
\begin{itemize}
\itemsep=0pt \parskip=0pt
  \item 整数除法只保留整数部分。
  \item 给定逆波兰表达式总是有效的。换句话说,表达式总会得出有效数值且不存在除数为 0 的情况。
\end{itemize}

\begin{tips}
使用题目给的提示:适合用栈操作运算:遇到数字则入栈;遇到算符则取出栈顶两个数字进行计算,并将结果压入栈中。
\end{tips}

\begin{lstlisting}[title=逆波兰表达式求值]
class Solution {
    private boolean isNumber(String str){
        return !"+".equals(str) && !"-".equals(str) && !"*".equals(str) && !"/".equals(str);
    }

    public int evalRPN(String[] tokens) {
        Deque<Integer> stk = new ArrayDeque<>();
        for(String str : tokens){
            if(isNumber(str)){
                stk.push(Integer.valueOf(str));
                continue;
            }
            int n1 = stk.pop();
            int n2 = stk.pop();
            switch(str){
                case "+":
                    stk.push(n2 + n1);
                    break;
                case "-":
                    stk.push(n2 - n1);
                    break;
                case "*":
                    stk.push(n2 * n1);
                    break;
                case "/":
                    stk.push(n2 / n1);
                    break;
            }

        }
        return stk.pop();
    }
}
\end{lstlisting}

\num 给你一个整数数组\dm{nums},你需要找出一个\textbf{连续子数组},如果对这个子数组进行升序排序,那么整个数组都会变为升序排序。请你找出符合题意的 最短 子数组,并输出它的长度。\href{https://leetcode-cn.com/problems/shortest-unsorted-continuous-subarray/} {OJ链接-最短无序连续子数组}

\begin{tips}
1.如果数组长度为0或长度为1,则数组一定有序。

2.对于长度大于1的数组:

符合题意的任意一个应该为:有序1 + 无序 + 有序2。

升序排序,所以无序中的最大值不会超过有序2中的最小值,无序中的最小值不会小于有序1中的最大值。

使用\dm{lo}表示无序的左边界,使用\dm{ri}表示无序的右边界。

从左至右找无序的右边界:将已经扫描过的最大元素记为\dm{max},显然有序2的所有元素都应该大于等于\dm{max},如果某个元素\dm{nums[i]<max},说明\dm{nums[i]}不在有序2中,应该在无序中,把无序的右边界更新为\dm{i}。

从右至左找左边界:与找右边界逻辑逻辑一致,将已经扫描过的最小元素记为\dm{min},有序1的所有元素都应该小于等于\dm{min},如果某个元素\dm{nums[j]>min},说明\dm{nums[j]}不在有序1中,应该在无序中,把无序的左边界更新为\dm{j}。
\end{tips}

\begin{lstlisting}[title=最短无序连续子数组]
class Solution {
    public int findUnsortedSubarray(int[] nums) {
        if(nums.length == 0 || nums.length == 1 ){
            return 0;
        }
        int lo = nums.length - 1;
        int hi = 0;
        int max = nums[0];
        int min = nums[nums.length - 1];
        for(int i = 0; i < nums.length; i++){
            if(nums[i] < max){
                hi = i;
            }else{
                max = nums[i];
            }

            if(nums[nums.length - 1 - i] > min){
                lo = nums.length - 1 - i;
            }else{
                min = nums[nums.length - 1 - i];
            }
        }
        return hi - lo < 0 ? 0 : hi - lo + 1;
    }
}
\end{lstlisting}

\num 给定一组字符,使用\textcolor[rgb]{0.00,1.00,0.25}{原地算法}将其压缩。压缩后的长度必须始终小于或等于原数组长度。数组的每个元素应该是长度为1 的字符(不是\dm{int}整数类型)。在完成原地修改输入数组后,返回数组的新长度。\href{https://leetcode-cn.com/problems/string-compression/}{OJ链接-压缩字符串}

\begin{tips}
使用三个指针。

1.指针\dm{i}指向开头。第三个指针\dm{index}从头开始。

2.\dm{j}指向\dm{i}的下一个位置。

3.如果指针\dm{j}指向的字符等于指针\dm{i}指向的字符,指针\dm{j}向后移动,直到与指针\dm{i}指向的字符不相等(或者到字符串的末尾)停下来。

4.两种情况:\\
a.若\dm{j}在\dm{i}下一位停下来,表示\dm{i}指向的字符只有一个,不需要压缩,此时\dm{j - i}为1。\\
b.若\dm{j}在\dm{i}的后边位置停下来(超过1位),假设\dm{i}为0,\dm{j}为4,说明位置\dm{0、1、2、3}的字符都相等,即有连续\dm{j - i}个字符相等。

5.在\dm{index}处记录\dm{i}处的字符,然后向右移动一位,如果\dm{j - i}为1,不需要压缩,如果\dm{j - i}大于1,题目给的字符数组的长度最大为1000,可能出现连续字符个数最大为1000,即\dm{j - i}超过一位数,把该数转换为字符数组,然后把数的每一位依次填入\dm{index}处(填一位\dm{index}向右移动一位)(代码的12-15行)。

6.上一步执行完后,将指针\dm{i}移到指针\dm{j}处,执行2-5步,直到全部字符串扫描结束。
\end{tips}

\begin{lstlisting}[title=压缩字符串]
class Solution {
    public int compress(char[] chars) {
        int index = 0;
        int i = 0;
        int j = 1;
        while(i < chars.length && index < chars.length){
            while(j < chars.length && chars[j] == chars[i]){
                j++;
            }
            chars[index++] = chars[i];
            int sum = j - i;
            if(sum > 1 && index < chars.length){
                for(char ch : ("" + sum).toCharArray()){
                    chars[index++] = ch;
                }
            }
            i = j;
            j = i + 1;
        }
        return index;
    }
}
\end{lstlisting}

\num 给定一个字符串,验证它是否是回文串,只考虑字母和数字字符,可以忽略字母的大小写。

说明:本题中,我们将空字符串定义为有效的回文串。
\href{https://leetcode-cn.com/problems/valid-palindrome/}{OJ链接-验证回文串}

\begin{tips}
1.处理空字符串:是回文串

2.只关注字母和数字,所以写一个私有方法来判断当前字符是否为字母/数字。

3.使用左右指针指向字符串的开头和结尾。

4.如果不是字母或数字,左指针右移,右指针左移。

5.经过上一后左右指针指向的字符则为字母或数字。题目中不区分大小写,所以如果字符是大写字母,将其转换为小写字母(也可以把小写的字符转为大写的)。

6.因为字符串是不可变对象,所以使用字符来接收左右指针指向的字符,大写字符转小写:\dm{ch - 'A' + 'a'},字符相加减本质上是其对应的\dm{ASCII}码相加减,因此还需将其强制转换为字符。

7.大写转为小写后,比较两个字符是否相等,如果不相等,返回\dm{false}。如果相等,左指针左移,右指针右移。继续第5到第7步。

8.若左指针和右指针重合,说明是回文串。
\end{tips}

\begin{lstlisting}[title=验证回文串]
class Solution {
    private boolean isAlphaAndNum(char ch){
        if((ch >= 'a' && ch <= 'z') || (ch >= 'A' && ch <= 'Z') || (ch >= '0' && ch <= '9')){
            return true;
        }
        return false;
    }
    public boolean isPalindrome(String s) {
        if(s == null || " ".equals(s)){
            return true;
        }
        int lo = 0;
        int hi = s.length() - 1;
        char a;
        char b;
        while(lo < hi){
            if(!isAlphaAndNum(s.charAt(lo))){
                lo++;
            }
            if(!isAlphaAndNum(s.charAt(hi))){
                hi--;
            }
            if(lo < hi && isAlphaAndNum(s.charAt(lo)) && isAlphaAndNum(s.charAt(hi))){
                a = s.charAt(lo);
                b = s.charAt(hi);
                if(a >= 'A' && a <= 'Z'){
                    a = (char)(a + 'a' - 'A');
                }
                if(b >= 'A' && b <= 'Z'){
                    b = (char)(b + 'a' - 'A');
                }
                if(a != b){
                    return false;
                }
                lo++;
                hi--;
            }
        }
        return true;
    }
}
\end{lstlisting}

\num 字符串转换函数\href{https://leetcode-cn.com/problems/string-to-integer-atoi/} {OJ链接-字符串转换整数atoi}。

请你来实现一个\dm{myAtoi(string s)}函数,使其能将字符串转换成一个 32 位有符号整数(类似\dm{C/C++}中的\dm{atoi}函数)。

函数\dm{myAtoi(string s)}的算法如下:
\begin{itemize}
\itemsep=0pt \parskip=0pt%删除排序列表之间的垂直空白
  \item 读入字符串并丢弃无用的前导空格。
  \item 检查第一个字符(假设还未到字符末尾)为正还是负号,读取该字符(如果有)。 确定最终结果是负数还是正数。 如果两者都不存在,则假定结果为正。
  \item 读入下一个字符,直到到达下一个非数字字符或到达输入的结尾。字符串的其余部分将被忽略。
  \item 将前面步骤读入的这些数字转换为整数(即,"123" $\rightarrow$ 123, "0032" $\rightarrow$32)。如果没有读入数字,则整数为 0 。必要时更改符号(从步骤 2 开始)。
  \item 如果整数数超过 32 位有符号整数范围$[−2^{31},  2^{31} − 1]$ ,需要截断这个整数,使其保持在这个范围内。具体来说,小于$−2^{31}$ 的整数应该被固定为$−2^{31}$ ,大于$2^{31} − 1$的整数应该被固定为$2^{31} − 1$。
\item 返回整数作为最终结果。
\end{itemize}

\textbf{注意}:
\begin{itemize}
\itemsep=0pt \parskip=0pt
\item 本题中的空白字符只包括空格字符\dm{' '} 。
\item 除前导空格或数字后的其余字符串外,请勿忽略 任何其他字符。
\end{itemize}

\begin{tips}
1.对于空字符串和\dm{null},直接返回0;

2.一般字符串:

依题意,首先扫描空格,当指针没有到达末尾且指向的字符为空格时,指针向右移动。

空格扫描完后,判断指针是否到达末尾,如果到达末尾,返回0.

没有到达末尾,扫描下一个字符是否是\dm{+}或\dm{-},使用\dm{flag}作为正负的标记(默认为1),若扫描到\dm{+},\dm{flag}不变,指针向后移动一位。若扫描到\dm{1},\dm{flag}置为\dm{-1},指针向后移动一位。

再次判断是否到达末尾,如果到达末尾,返回0.

没有到达末尾,继续扫描:(使用\dm{ret}存储扫描的数字,因为数字可能超过\dm{int}的最大范围,所以\dm{ret}为\dm{long}类型)

如果是数字,将其放到\dm{ret}的后边,即\dm{ret = ret * 10},然后判断\dm{ret}的大小,如果\dm{flag}为1,且\dm{ret}超过整型的最大值,则返回\dm{Integer.MAX_VALUE},如果\dm{flag}为-1,且\dm{-ret}小于整型的最小值,则返回\dm{Integer.MIN_VALUE}。

如果不是数字或者指针已经扫描到了末尾,退出循环。

如果\dm{ret}没有超过整型的范围,返回\dm{flag * ret}。要注意类型的转换。
\end{tips}

\begin{lstlisting}[title=字符串转换整数atoi]
class Solution {
    public int myAtoi(String s) {
        if(s == null || s.length() == 0){
            return 0;
        }
        int i = 0;
        long ret = 0;
        int flag = 1;
        while(i < s.length() && s.charAt(i) == ' '){
            ++i;
        }
        if(i >= s.length()){
            return 0;
        }
        if(s.charAt(i) == '-'){
            flag = -1;
            i++;
        }else if(s.charAt(i) == '+'){
            i++;
        }

        if(i >= s.length()){
            return 0;
        }

        while(i < s.length() && (s.charAt(i) >= '0' && s.charAt(i) <= '9')){
            int x = s.charAt(i) - '0';
            ret = ret * 10 + x;
            if(flag > 0 && ret > Integer.MAX_VALUE){
                return Integer.MAX_VALUE;
            }
            if(flag < 0 && -ret < Integer.MIN_VALUE){
                return Integer.MIN_VALUE;
            }
            ++i;
        }
        return flag > 0 ? (int)ret : -(int)ret;
    }
}
\end{lstlisting}

\num 给你两个二进制字符串,返回它们的和(用二进制表示)。
输入为 \textbf{非空} 字符串且只包含数字 \dm{1} 和 \dm{0}。\href{https://leetcode-cn.com/problems/add-binary/} {OJ链接-二进制求和}

\begin{tips}
1.首先将短的字符串用0将高位补齐。因为要对字符频繁的操作,所以使用可变字符序列对象\dm{StringBuilder}。可以先将题中的两个字符串翻转,在最短的字符串后补0(使用\dm{append}方法),最后计算结束再将字符串翻转。

2.字符串补齐后,使用列竖式的方法计算每一位,用\dm{carry}来表示进位,刚开始进位为0,。

3.获取每一位的字符,因为只有0和1,所以减去字符0即为数字0和1。每一位的和为\dm{carry + a对应位置的值 + b对应位置的值},记为\dm{count},因为是2进制,所以\dm{count}大于等于2时,要进行进位,即\dm{carry}置为1,同时该位置的\dm{count\%2}。

4.计算结束后,要判断\dm{carry}的值,如果这个值是1,说明还需要进一位。

5.别忘了翻转和返回值的类型。
\end{tips}

\begin{lstlisting}[title=二进制求和]
class Solution {
    public String addBinary(String a, String b) {
        //求两个字符串的长度
        int a1 = a.length();
        int b1 = b.length();
        //求字符串的最大长度
        int max = a1 >= b1 ? a1 : b1;
        //将字符串存入可变字符序列并进行翻转
        StringBuilder stringBuilder1 = new StringBuilder(a).reverse();
        StringBuilder stringBuilder2 = new StringBuilder(b).reverse();
        //对短字符串进行补0
        while(stringBuilder1.length() < max){
            stringBuilder1.append("0");
        }
        while(stringBuilder2.length() < max){
            stringBuilder2.append("0");
        }
        //进位,初始为0
        int carry = 0;

        int x = 0;
        int y = 0;
        int count = 0;
        //存放相加的结果
        StringBuilder ret = new StringBuilder();
        for(int i = 0; i < max; i++){
            x = stringBuilder1.charAt(i) - '0';
            y = stringBuilder2.charAt(i) - '0';
            //每一位结果等于进位加各数
            count = x + y + carry;
            //二进制,结果超过2,就需要进位
            if(count >= 2){
                carry = 1;
                ret.append(count % 2);
            }else{
                carry = 0;
                ret.append(count);
            }
        }
        //最高位的进位判断
        if(carry == 1){
            ret.append("1");
        }
        //别忘了翻转,注意方法的返回类型
        return ret.reverse().toString();
    }
}
\end{lstlisting}

\num 给定一个整数数组 nums 和一个整数目标值 target,请你在该数组中找出 和为目标值 的那 两个 整数,并返回它们的数组下标。
你可以假设每种输入只会对应一个答案。但是,数组中同一个元素不能使用两遍。
你可以按任意顺序返回答案。要求时间复杂度$O(n)$,当然也可以选择使用暴力的$O(n^2)$的解法。\href{https://leetcode-cn.com/problems/two-sum/} {OJ链接-两数之和}

\begin{tips}
1.从左至右遍历数组。

2.一个指针指向当前元素,另一个指针从当前元素的下一位至数组最后遍历,寻找与当前元素的和等于目标值的元素。找到后返回这两个指针对应的下标值。
\end{tips}

\begin{lstlisting}[title=两数之和]
class Solution {
    public int[] twoSum(int[] nums, int target) {
        for(int i = 0; i< nums.length;i++){
            for(int j = i + 1;j < nums.length;j++){
                if(nums[j] == target - nums[i]){
                    int[] ret = {i,j};
                    return ret;
                }
            }
        }
        return new int[0];
    }
}
\end{lstlisting}

\num 给你一个非空数组,返回此数组中 第三大的数 。如果不存在,则返回数组中最大的数。\href{https://leetcode-cn.com/problems/third-maximum-number/} {OJ链接-第三大的数}

\begin{tips}
1.数组长度为1,直接返回该元素。

2.数组长度为2,返回两个中最大的那个。

3.数组长度大于等于3:

题目中数的范围为\dm{int}的范围,所以数组中第三大的元素可能是\dm{int}的最小值(\dm{Integer.MIN_VALUE})。

使用三个数\dm{max1}、\dm{max2}、\dm{max3}分别表示数组中第一大,第二大,第三大的元素,将其初始化为\dm{int}的最小值减一(即\dm{Integer.MIN_VALUE - 1},把这个值记为\dm{min}),因为超出了\dm{int}的范围,所以;类型为\dm{long}。

遍历数组,对于任意元素:

如果该元素大于\dm{max1},就把\dm{max2}的值赋给\dm{max3},把\dm{max1}的值赋给\dm{max2},把该元素的值赋给\dm{max1};

如果该元素等于\dm{max1},检查下一个元素,即\dm{continue};

如果该元素小于\dm{max1},则将该元素与\dm{max2}比较:

如果该元素大于\dm{max2},就把\dm{max2}的值赋给\dm{max3},把该元素的值赋给\dm{max2};

如果该元素等于\dm{max2},检查下一个元素,即\dm{continue};

如果该元素小于\dm{max2},则将该元素与\dm{max3}比较:

只有该元素大于\dm{max3}时,才将该元素的值赋给\dm{max3},否则不操作。

遍历完数组后,看\dm{max3}的值是否为\dm{min},如果是,说明没有第3大的数,返回\dm{max1},否则返回\dm{max3}。因为其类型为\dm{long},还需要进行类型转换。
\end{tips}

\begin{lstlisting}[title=第三大的数]
class Solution {
    public int thirdMax(int[] nums) {
        if(nums.length == 1){
            return nums[0];
        }
        if(nums.length == 2){
            return nums[0] > nums[1] ? nums[0] : nums[1];
        }
        long min = (long)Integer.MIN_VALUE - 1;
        long max1 = min;
        long max2 = max1;
        long max3 = max2;

        for(int i = 0; i < nums.length; i++){
            if(nums[i] > max1){
                max3 = max2;
                max2 = max1;
                max1 = nums[i];
            }else if(nums[i] == max1) {
                continue;
            }else{
                if(nums[i] > max2){
                    max3 = max2;
                    max2 = nums[i];
                }else if(nums[i] == max2){
                    continue;
                }else{
                    if(nums[i] > max3){
                        max3 = nums[i];
                    }
                }
            }
        }
        return max3 == min ? (int)max1 : (int)max3;
    }
}
\end{lstlisting}



\num 给定一个由\textbf{整数}组成的 \textbf{非空} 数组所表示的非负整数,在该数的基础上加一。最高位数字存放在数组的首位, 数组中每个元素只存储单个数字。你可以假设除了整数 0 之外,这个整数不会以零开头。(第一个数字是9或者最后一个是9要考虑进位)\href{https://leetcode-cn.com/problems/plus-one/} {OJ链接-加一}

\begin{tips}
1.将数组的最后一个元素加一;

2.加一后只有两种情况,等于10或小于10.

3.等于10,将其除10取余即为该位的值,同时将指针前移一位,将前一位加一,然后重复3、4步;

4.小于10,不需要进位,直接返回。

5.对于9,99,999…这种加一后,长度增加的,直接创建一个比原数组长度长1的数组,将新数组首位初始化为1,其余保持不变(默认初始化为0)。
\end{tips}


\begin{lstlisting}[title=加一]
class Solution {
    public int[] plusOne(int[] digits) {
        for(int i = digits.length - 1; i >= 0; i--){
            digits[i]++;
            digits[i] = digits[i] % 10;
            if(digits[i] != 0){
                return digits;
            }
        }
        int[] arr = new int[digits.length + 1];
        arr[0] = 1;
        return arr;
    }
}
\end{lstlisting}

\num 给定一个整数类型的数组\dm{ nums},请编写一个能够返回数组 “中心索引” 的方法。

我们是这样定义数组 中心索引 的:数组中心索引的左侧所有元素相加的和等于右侧所有元素相加的和。

如果数组不存在中心索引,那么我们应该返回 -1。如果数组有多个中心索引,那么我们应该返回最靠近左边的那一个。\href{https://leetcode-cn.com/problems/find-pivot-index/} {OJ链接-寻找数组的中心索引}

\begin{tips}
1.计算出数组的和;

2.从左至右遍历数组,计算出第\dm{i}个元素左侧元素的和(包含该元素\dm{A[i]})\dm{sum1}和右侧元素的和(包含该元素\dm{A[i]})\dm{sum2},如果\dm{sum1}和\dm{sum2}相等,说明该位置为中心索引。

3.遍历完依旧没有符合题意的返回\dm{-1}。
\end{tips}

\begin{lstlisting}[title=寻找数组的中心索引]
class Solution {
    public int pivotIndex(int[] nums) {
        int sum1 = 0;
        for(int x : nums){
            sum1 += x;
        }
        int sum2 = 0;
        for(int i = 0; i < nums.length; i++){
            sum2 += nums[i];
            if(sum1 == sum2){
                return i;
            }
            sum1 = sum1 - nums[i];
        }
        return -1;
    }
}
\end{lstlisting}

\num 给定一个非负的整数数组\dm{A},返回一个数组,在该数组中, \dm{A}的所有偶数元素之后跟着所有奇数元素。
\href{https://leetcode-cn.com/problems/sort-array-by-parity/} {OJ链接-按奇偶排序数组}

\begin{tips}
1.使用左指针指向数组的左侧,右指针指向数组的右侧

2.如果左指针指的是偶数,左指针向右移动,右指针指的是奇数,右指针向左移动

3.经过上一步,左指针指向为奇数,右指针指向的为偶数,交换这两个指针指的元素的值。

4.左指针在右指针的左侧时,重复第二步、第三部。
\end{tips}

\begin{lstlisting}[title=按奇偶排序数组]
class Solution {
    public int[] sortArrayByParity(int[] A) {
        int lo = 0;
        int hi = A.length - 1;
        while(lo < hi){
            if(A[lo] % 2 == 0){
                lo++;
            }
            if(A[hi] % 2 == 1){
                hi--;
            }
            if(lo < hi && A[lo] % 2 == 1 && A[hi] % 2 == 0){
                int tmp = A[lo];
                A[lo] = A[hi];
                A[hi] = tmp;
            }
        }
        return A;
    }
}
\end{lstlisting}

\num 给定一个字符串 S,返回 “反转后的” 字符串,其中不是字母的字符都保留在原地,而所有字母的位置发生反转。\href{https://leetcode-cn.com/problems/reverse-only-letters/} {OJ链接-仅仅反转字母}

\begin{tips}
1.使用两个指针分别指向字符串的前后。

2.判断左右指针指的元素是否为字母,如果左指针指的不是字母,左指针向右移动,右指针指的不是字母,右指针向左移动。

3.经过第二步后,两个指针指的为字母,如果两个指针指的字母相等,不用交换,如果不相等,交换,交换完毕,左指针向右移动一下,右指针向左移动一下。第二步第三部。

4.如果左指针没有越过右指针,或左指针和右指针重合之前,重复
\end{tips}

\begin{lstlisting}[title=仅仅反转字母]
class Solution {
    private boolean isAlpha(char ch){
        return ((ch >= 'a' && ch <= 'z') || (ch >= 'A' && ch <= 'Z'));
    }
    public String reverseOnlyLetters(String S) {
        int lo = 0;
        int hi = S.length() - 1;
        char[] ch = S.toCharArray();
        while(lo < hi){
            if(!isAlpha(S.charAt(lo))){
                lo++;
            }
            if(!isAlpha(S.charAt(hi))){
                hi--;
            }
            if(lo < hi && isAlpha(S.charAt(lo)) && isAlpha(S.charAt(hi))){
                if(S.charAt(lo) != S.charAt(hi)){
                    char tmp = ch[lo];
                    ch[lo] = ch[hi];
                    ch[hi] = tmp;
                }
                lo++;
                hi--;
            }
        }
        return new String(ch);
    }
}
\end{lstlisting}

\num 给定一个按非递减顺序排序的整数数组\dm{A},返回每个数字的平方组成的新数组,要求新数组也按非递减顺序排序。(注意:非递减顺序即递增,要注意原数组里的负数)\href{https://leetcode-cn.com/problems/squares-of-a-sorted-array/} {OJ链接-有序数组的平方}

\begin{tips}
1.类似于归并排序,使用两个指针指向数组的左边和右边。

2.判断两个指针指向的元素的平方的大小,将大的那个的平方填入新数组的后边,即如果左指针指向的元素的平方大,就把该元素平方填到新数组的后边,然后左指针向右移动一位,新数组中的指针向前移动一位,以此类推。

3.退出条件:左指针越过右指针,或者新数组的指针小于0.两者是等价的
\end{tips}

\begin{lstlisting}[title=有序数组的平方]
class Solution {
    public int[] sortedSquares(int[] nums) {
        int[] arr = new int[nums.length];
        int lo = 0;
        int hi = nums.length - 1;
        int i = nums.length - 1;
        while(lo <= hi){//或者写成while(i >= 0)
            if(nums[lo] * nums[lo] < nums[hi] * nums[hi]){
                arr[i--] = nums[hi] * nums[hi];
                hi--;
            }else{
                arr[i--] = nums[lo] * nums[lo];
                lo++;
            }
        }
        return arr;
    }
}
\end{lstlisting}

\num
给你两个有序整数数组 \dm{nums1}和\dm{nums2},请你将\dm{nums2}合并到\dm{nums1}中,使\dm{nums1}成为一个有序数组。\href{https://leetcode-cn.com/problems/merge-sorted-array/} {OJ链接-合并两个有序数组}

\begin{tips}
方法1:将两个数组合并,然后重新排序。

方法2:依据题目描述,此题为归并排序算法的归并部分。

因为最终的结果要存放在\dm{nums1}中,为了避免\dm{nums1}中的数据被覆盖,使用一个数组\dm{arr}来存放\dm{nums1}的数据。

步骤:

1.使用两个指针分别指向数组\dm{arr}和\dm{nums2}的开头,比较其对应元素的大小,将较小的那个元素放入到\dm{nums1}中,同时将\dm{nums1}和拿出元素的那个数组的指针向后移动一位。

2.其中一个数组遍历完后,将没有遍历完的那个数组的剩余部分放在\dm{nums1}后边即可。
\end{tips}

\begin{lstlisting}[title=合并两个有序数组]
class Solution {
    public void merge(int[] nums1, int m, int[] nums2, int n) {
        int[] arr = new int[m];
        for(int i = 0; i < m; i++){
            arr[i] = nums1[i];
        }
        int a1 = 0;
        int a2 = 0;
        int i = 0;
        while(a1 < m && a2 < n){
            if(arr[a1] <= nums2[a2]){
                nums1[i++] = arr[a1++];
            }else{
                nums1[i++] = nums2[a2++];
            }
        }
        for(;i < m + n; i++){
            if(a1 < m){
                nums1[i] = arr[a1++];
            }
            if(a2 < n){
               nums1[i] = nums2[a2++];
            }
        }
    }
}
\end{lstlisting}

\num 给定一个仅包含大小写字母和空格\dm{' '}的字符串,返回其最后一个单词的长度。如果不存在最后一个单词,请返回0。\href{https://leetcode-cn.com/problems/length-of-last-word/} {最后一个单词的长度}

\begin{tips}
1.字符串长度为0或者字符串引用指向为空,那最后一个单词长度为0.

2.一般情况:

用右指针\dm{hi}从字符串的最后一位开始遍历,如果是空格,指针就向前移动,如果遇到了字母就停下来,记录该位置。当\dm{hi}小于0时,说明字符串遍历完毕也没遇到字母,即不存在最后一个单词,返回0.

如果\dm{hi}不小于0,记录\dm{hi}当前的位置,使用一个新的指针\dm{lo}从该位置向前遍历,直到遇到空格(或遍历字符串结束即\dm{lo}小于0)停下来,\dm{lo}和\dm{hi}之间的字符个数就是最后一个单词的长度。
\end{tips}

\begin{lstlisting}[title=最后一个单词的长度]
class Solution {
    public int lengthOfLastWord(String s) {
        //char[] ch = s.toCharArray();
        int hi = s.length() - 1;
        if(s.length() == 0 || s == null){
            return 0;
        }
        while(hi >= 0 && s.charAt(hi) == ' '){
            hi--;
        }
        if(hi < 0){
            return 0;
        }
        int lo = hi;
        while(lo >= 0 && s.charAt(lo) != ' '){
            lo--;
        }
        return hi - lo;
    }
}
\end{lstlisting}


\num \href{https://leetcode-cn.com/problems/ransom-note/} {赎金信}

给定一个赎金信 (\dm{ransom}) 字符串和一个杂志(\dm{magazine})字符串,判断第一个字符串\dm{ransom}能不能由第二个字符串\dm{magazines}里面的字符构成。如果可以构成,返回\dm{true};否则返回\dm{false}。

(题目说明:为了不暴露赎金信字迹,要从杂志上搜索各个需要的字母,组成单词来表达意思。杂志字符串中的每个字符只能在赎金信字符串中使用一次。)

假设两个字符串均只含有小写字母。

\begin{tips}
1.杂志字符串长度小于赎金信字符串长度,一定不能构成赎金信。

2.因为只有小写字母,所以分别统计各字符的个数,将两字符串各字符的个数放入数组中。

3.分别比较两字符串对应字符的个数,赎金信相应字符的个数超过杂志该字符的个数,则一定不能表示,否则就能表示。
\end{tips}

\begin{lstlisting}[title=赎金信]
class Solution {
    public boolean canConstruct(String ransomNote, String magazine) {
        if(ransomNote.length()>magazine.length()){
            return false;
        }
        char[] ch1 = ransomNote.toCharArray();
        char[] ch2 = magazine.toCharArray();
        int[] rans = new int[26];
        int[] maga = new int[26];
        for(char c1 : ch1){
            rans[c1 - 'a']++;
        }
        for(char c2 : ch2){
            maga[c2 - 'a']++;
        }
        for(int i = 0; i < 26; i++){
            if(rans[i] > maga[i]){
                return false;
            }
        }
        return true;
    }
}
\end{lstlisting}

\num \href{https://leetcode-cn.com/problems/palindrome-number/\%20--with-chrome-plus-plus\%20--disable-features=RendererCodeIntegrity,FlashDeprecationWarning/} {回文数}

\begin{tips}
1.根据题意,所有负数都不是回文数。

2.小于10的自然数都是回文数。

3.能被10整除的都不是回文数。(能被10整除,最后一位是0,而第一位不会是0,所以不是回文数)

4.其他的数字\dm{x},每次从后边取一位\dm{x\%10},放入新的数\dm{num}中:\dm{num = num * 10 + x\%10},

5.每取完一位,就将最后一位拿掉:\dm{x = x / 10}。

判断\dm{x}是不是小于\dm{num},如果小于,说明已经过半了,即:

6.如果\dm{x}是奇数位,例如5位,取了2位后,\dm{x}还有3位,此时\dm{num}是2位,\dm{num < x}。再取一位,\dm{x}变成2位,\dm{num}变成3位,两位数小于三位数,此时说明取的位数超过一半,就不用取了,如果\dm{x}和\dm{num}的前两位\dm{num \ 10}相等,说明是回文数,否则不是。

7.如果\dm{x}是偶数位,例如4位,取了2位后,\dm{num}是2位,\dm{x}剩余2位,如果\dm{num}等于\dm{x},说明是回文数。
\end{tips}

\begin{lstlisting}[title=取一半数字进行判断]
class Solution {
    public boolean isPalindrome(int x) {
        if(x < 0){
            return false;
        }
        if(0 <= x && x < 10){
            return true;
        }
        if(x % 10 == 0){
            return false;
        }
        int num = 0;
        while(x > num){
            num = num * 10 + x % 10;
            x /= 10;
        }
        return x == num || x == num / 10;
    }
}
\end{lstlisting}


\num \href{https://leetcode-cn.com/problems/search-insert-position/\%20--with-chrome-plus-plus\%20--disable-features=RendererCodeIntegrity,FlashDeprecationWarning/} {搜索插入位置}

\begin{tips}
有序数组,采用二分查找。

1.找到元素:返回元素的下标。

2.如果数组中没有该元素。退出循环的前一次左指针、右指针、中间指针指向同一元素,若该元素小于目标值,即\dm{nums[mid] < target},左指针指向中间指针的下一个元素,此时左指针在右指针的右边,退出循环。同时左指针指向的位置也是目标元素要插入的位置。

3.若该元素大于目标值,即\dm{target < nums[mid]},右指针指向中间指针的前一个元素,右指针在左指针的左边,退出循环。左指针的左侧元素必定小于目标值,而左指针指向的元素大于目标值,目标值应该插入到左指针指向的位置处。

4.综上,找到目标值就返回该目标值对应的下标,没找到该目标值就返回左指针指向的位置。
\end{tips}

\begin{lstlisting}
class Solution {
    public int searchInsert(int[] nums, int target) {
        if(nums.length == 0){
            return 0;
        }
        int lo = 0;
        int hi = nums.length - 1;
        int mid = (lo + hi)/2;
        while(lo <= hi){
            mid = (lo + hi)/2;
            if(nums[mid] < target){
                lo = mid + 1;
            }else if(target < nums[mid]){
                hi = mid - 1;
            }else{
                return mid;
            }
        }
        return lo;
    }
}
\end{lstlisting}

\num \href{https://leetcode-cn.com/problems/remove-element/\%20--with-chrome-plus-plus\%20--disable-features=RendererCodeIntegrity,FlashDeprecationWarning/} {移除元素}

\begin{tips}
1.边界情况:数组为空时,要返回0。

分析:不考虑顺序,返回的为不含\dm{val}的前边一段,思路:把\dm{val}换到后边去。

2.左指针\dm{lo}指向数组的左边,找\dm{val},如果没找到\dm{val},就把这个指针向右移动;

3.右指针\dm{hi}指向数组的右边,\dm{val}就应该位于数组后边,所以如果值是\dm{val},指针就左移。

4.经过2、3步后,左指针指的是\dm{val},右指针指的是非\dm{val},将这俩指针对应元素交换。

5.左右指针重合时,数组就遍历完了。

6.左指针的左边一定是不含\dm{val}的,其左边有\dm{lo}个元素,如果此时的左指针对应元素的值时\dm{val},那就返回\dm{lo},如果左指针对应元素的值不是\dm{lo},就返回\dm{lo + 1}。
\end{tips}

\begin{lstlisting}
class Solution {
    public int removeElement(int[] nums, int val) {
        if(nums.length == 0){
            return 0;
        }
        int lo = 0;
        int hi = nums.length - 1;
        int len = 0;
        while(lo < hi){
            if(nums[lo] != val && lo < hi){
                lo++;
            }
            if(nums[hi] == val && lo < hi){
                hi--;
            }
            if(lo < hi){
                int tmp = nums[lo];
                nums[lo] = nums[hi];
                nums[hi] = tmp;
            }
        }
        if(nums[lo] == val){
            return lo;
        }
        return lo + 1;
    }
}
\end{lstlisting}


\num \href{https://leetcode-cn.com/problems/rotate-array/\%20--with-chrome-plus-plus\%20--disable-features=RendererCodeIntegrity,FlashDeprecationWarning/} {旋转数组}

\begin{tips}
在C语言做过这个题。

1.要先把旋转的位置除数组的长度取余,因为向右移动数组长度个位置(或其倍数)等于没有移动。

2.把整个数组旋转得到新的数组。

3.依次旋转数组的前\dm{k}个元素(\dm{0到k-1}),和后边剩余的元素。
\end{tips}

\begin{lstlisting}
class Solution {
    public void rotate(int[] nums, int k) {
        k %= nums.length;
        reverse(nums,0,nums.length-1);
        reverse(nums,0,k-1);
        reverse(nums,k,nums.length-1);

    }
    public void reverse(int[] nums,int lo,int hi){
        while(lo < hi){
            int tmp = nums[lo];
            nums[lo] = nums[hi];
            nums[hi] = tmp;
            lo++;
            hi--;
        }
    }
}
\end{lstlisting}




\num \href{https://leetcode-cn.com/problems/to-lower-case/\%20--with-chrome-plus-plus\%20--disable-features=RendererCodeIntegrity,FlashDeprecationWarning/} {转换成小写字母}

\begin{tips}
转大写:\dm{str.toUpperCase()}

转小写:\dm{str.toLowerCase()}
\end{tips}

\begin{lstlisting}[title= 转换成小写字母]
class Solution {
    public String toLowerCase(String str) {
        return str.toLowerCase();
    }
}
\end{lstlisting}



\num \href{https://leetcode-cn.com/problems/reverse-linked-list/}{反转一个单链表}

\begin{lstlisting}[title = 示例]
输入: 1->2->3->4->5->NULL
输出: 5->4->3->2->1->NULL
\end{lstlisting}

\begin{tips}
1.链表为空(\dm{head == null})或者链表只有一个元素(\dm{head.next == null})。反转后还是自身:\dm{return head;}

2.链表反转:首先用一个引用储存当前元素的下一个元素。\dm{nextNode = curNode;}然后把当前元素\dm{curNode}的\dm{next}指向它的前一个元素\\ \dm{prevNode}:\dm{curNode.next = prevNode}

3.第2步执行完后,处理下一个元素:\dm{prevNode = curNode;curNode = curNode.next}。

4.\dm{curNode == null}时,说明到了链表的末端(循环退出条件)。此时\dm{prevNode}指向的是原链表的最后一个元素,即新链表的首节点:\\ \dm{return prevNode;}

5.初始化:第一个元素没有前驱结点,反转后,原链表第一个节点为新链表的最后一个节点,新链表的最后一个节点的\dm{next}应指向\dm{null}。所以\dm{prevNode}初始化为\dm{null}:\dm{Node prevNode = null;},\dm{curNode}最开始应指向链表的首节点:\dm{Node curNode = head;}。
\end{tips}

\begin{lstlisting}[title = 迭代版本]
/**
 * Definition for singly-linked list.
 * public class ListNode {
 *     int val;
 *     ListNode next;
 *     ListNode() {}
 *     ListNode(int val) { this.val = val; }
 *     ListNode(int val, ListNode next) { this.val = val; this.next = next; }
 * }
 */
class Solution {
    public ListNode reverseList(ListNode head) {
        if(head == null || head.next == null){
            return head;
        }
        ListNode prevNode = null;
        ListNode curNode = head;

        while(curNode != null){
            ListNode nextNode = curNode.next;
            curNode.next = prevNode;
            prevNode = curNode;
            curNode = nextNode;
        }
        return prevNode;
    }
}
\end{lstlisting}

\begin{tips}
1.可以将问题分为一个头结点和链表的剩余部分。

2.头结点为问题的平凡解:链表只有一个节点,直接返回。

3.子问题的解:头节点(\dm{head})的下一个节点\dm{head.next}的\dm{next}指向头结点。

4.头结点指向\dm{null}
\end{tips}

\begin{lstlisting}[title = 递归版本]
class Solution {
    public ListNode reverseList(ListNode head) {
        if(head == null || head.next == null){
            return head;
        }
        ListNode p = reverseList(head.next);
        head.next.next = head;
        head.next = null;
        return p;
    }
}
\end{lstlisting}

\num \href{https://leetcode-cn.com/problems/remove-linked-list-elements/}{移除链表元素}

删除链表中等于给定值 val 的所有节点。
\begin{lstlisting}[title = 示例]
输入: 1->2->6->3->4->5->6, val = 6
输出: 1->2->3->4->5
\end{lstlisting}

\begin{tips}
要考虑链表为空的情况。链表为空,直接返回\dm{null}

使用傀儡节点\dm{dummy}来进行辅助删除。\dm{dummy.next = head}

两个指针,一个指向待删除元素(\dm{cur})。一个指向待删除元素的前一个位置(\dm{prev})。

如果找到了待删除元素:将该元素前一个位置(\dm{pre})的\dm{next}指向删除元素的\dm{next}:\dm{pre.next = cur.next}。然后将\dm{cur}复位,即指向\dm{pre.next}:\dm{cur = pre.next}

如果没有找到,两个指针同时向后移动:\dm{pre = pre.next;cur = cur.next;}

循环结束的条件:\dm{cur}指向空,表示已经找到了链表的最后。

最后返回\dm{dummy.next}
\end{tips}

\begin{lstlisting}
/**
 * Definition for singly-linked list.
 * public class ListNode {
 *     int val;
 *     ListNode next;
 *     ListNode() {}
 *     ListNode(int val) { this.val = val; }
 *     ListNode(int val, ListNode next) { this.val = val; this.next = next; }
 * }
 */
class Solution {
    public ListNode removeElements(ListNode head, int val) {
        if (head == null){
            return null;
        }
        ListNode dummy = new ListNode(0,head);
        ListNode prev = dummy;
        ListNode cur = dummy.next;
        while(cur != null){
            if(cur.val == val){
                prev.next = cur.next;
                cur = prev.next;
            }else{
                prev = prev.next;
                cur = cur.next;
            }
        }
        return dummy.next;
    }
}
\end{lstlisting}

\num \href{https://leetcode-cn.com/problems/pascals-triangle/}{杨辉三角}

给定一个非负整数\dm{numRows},生成杨辉三角的前\dm{numRows}行。

\begin{lstlisting}
class Solution {
    public List<List<Integer>> generate(int numRows) {

        List<List<Integer>> pasTri = new ArrayList<>();
        if(numRows == 0){
            return pasTri;
        }
        List<Integer> first = new ArrayList<>();
        first.add(1);
        pasTri.add(first);
        if(numRows == 1){
            return pasTri;
        }
        List<Integer> second = new ArrayList<>();
        second.add(1);
        second.add(1);
        pasTri.add(second);
        if(numRows == 2){
            return pasTri;
        }
        for(int i = 3; i <= numRows; i++){
            List<Integer> list1 = new ArrayList<>();
            list1.add(1);
            for(int j = 2; j <= i - 1; j++){
                int a = pasTri.get(i - 1 - 1).get(j - 1);
                int b = pasTri.get(i - 1 - 1).get(j - 1 - 1);
                list1.add(a + b);
            }
            list1.add(1);
            pasTri.add(list1);
        }
        return pasTri;
    }
}
\end{lstlisting}

\num \href{https://leetcode-cn.com/problems/three-consecutive-odds/}{存在连续三个奇数的数组}

给你一个整数数组\dm{arr},请你判断数组中是否存在连续三个元素都是奇数的情况:如果存在,请返回\dm{true};否则,返回\dm{false}。
\begin{lstlisting}[title=示例1 ]
输入:arr = [2,6,4,1]
输出:false
解释:不存在连续三个元素都是奇数的情况。
\end{lstlisting}
\begin{lstlisting}[title=示例2]
输入:arr = [1,2,34,3,4,5,7,23,12]
输出:true
解释:存在连续三个元素都是奇数的情况,即 [5,7,23] 。
\end{lstlisting}

\begin{lstlisting}[title=版本1:LeetCode提供的]
class Solution {
    public boolean threeConsecutiveOdds(int[] arr) {
        for(int i = 0; i < arr.length - 2; i++){
            if((arr[i] % 2 == 1) && (arr[i+1] % 2 == 1) && (arr[i + 2] % 2 == 1)){
                return true;
            }
        }
        return false;
    }
}
\end{lstlisting}

\begin{lstlisting}[title=版本2:自己使用顺序表实现]
public class MyArrayList1 {
    public boolean threeConsecutiveOdds(List<Integer> arr) {
        for (int i = 0; i < arr.size() - 2; i++) {
            if ((arr.get(i) % 2 == 1) && (arr.get(i + 1) % 2 == 1) && (arr.get(i + 2) % 2 == 1)){
                return true;
            }
        }
        return false;
    }
}
\end{lstlisting}
\num \href{https://leetcode-cn.com/problems/employee-importance/}{员工的重要性}

给定一个保存员工信息的数据结构,它包含了员工唯一的id,重要度和直系下属的id。
比如,员工1是员工2的领导,员工2是员工3的领导。他们相应的重要度为15, 10, 5。那么员工1的数据结构是[1, 15, [2]],员工2的数据结构是[2, 10, [3]],员工3的数据结构是[3, 5, []]。注意虽然员工3也是员工1的一个下属,但是由于并不是直系下属,因此没有体现在员工1的数据结构中。

现在输入一个公司的所有员工信息,以及单个员工id,返回这个员工和他所有下属的重要度之和。

\begin{lstlisting}
示例 1:

输入: [[1, 5, [2, 3]], [2, 3, []], [3, 3, []]], 1
输出: 11
解释:
员工1自身的重要度是5,他有两个直系下属2和3,而且2和3的重要度均为3。因此员工1的总重要度是 5 + 3 + 3 = 11。
注意:
\end{lstlisting}
一个员工最多有一个直系领导,但是可以有多个直系下属,员工数量不超过2000。

\begin{tips}
1.遍历整个员工列表employees,找到符合id的员工employee

2.这个员工employee如果没有下属employee.subordinates.size()==0,重要度就是自己的重要度employee.importance。

3.如果这个员工有下属,算出每个下属及下属的重要度。
\end{tips}

\begin{lstlisting}
/*
// Definition for Employee.
class Employee {
    public int id;
    public int importance;
    public List<Integer> subordinates;
};
*/

class Solution {
    public int getImportance(List<Employee> employees, int id) {

        for(int i = 0;i < employees.size();i++){
            Employee employee = employees.get(i);
            if(employee.id == id){
                if(employee.subordinates.size() == 0){//没有下属
                    return employee.importance;
                }
                for(int j = 0; j < employee.subordinates.size(); j++){
                    employee.importance += getImportance(employees,employee.subordinates.get(j));
                }
                return employee.importance;
            }
        }
        return 0;
    }
}
\end{lstlisting}
\end{document} 