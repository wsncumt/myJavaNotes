\documentclass[a4paper]{report}
\usepackage[space,fancyhdr,fntef]{ctexcap}
\usepackage{fontspec}
\fontspec{宋体}
\setmainfont{Times New Roman}
%\fontsize{50pt}{50pt}\selectfont
\renewcommand{\rmdefault}{ptm}
\usepackage[namelimits,sumlimits,nointlimits]{amsmath}
\usepackage[text={169mm,250mm},bottom=20mm,top=25mm,left=25mm,right=15mm,centering]{geometry}
\usepackage{color}
\usepackage{CJKfntef}%下划线宏包160页
\usepackage{xcolor}
\usepackage{arydshln}%234页,虚线表格宏包
\pagestyle{fancy} \fancyhf{}
\fancyhead[OC]{\color{gray}\rightmark}

\fancyhead[EC]{\color{gray}\leftmark}
\fancyfoot[C]{\color{gray}\thepage}
\renewcommand{\headrule}{\color{gray}\hrule width\headwidth}
%\renewcommand{\footrulewidth}{0.4pt}%改为0pt即可去掉页脚上面的横线
%\usepackage{parskip}
%\usepackage{indentfirst}
\usepackage{graphicx}%插图宏包,参见手册318页
\definecolor{dkgreen}{RGB}{106,135,89}
\definecolor{dkblue}{RGB}{103,150,163}
\definecolor{wgray}{RGB}{248,248,248}
\definecolor{WGRAY}{RGB}{248,248,248}
\usepackage{listings}
\lstset{language=Java,
backgroundcolor=\color{wgray},
rulesepcolor=\color{red!20!green!20!blue!20},%代码块边框为淡青色
%lablestep=1,
%lablesep=5pt,
%lablestyle=\tiny,
%tablesize=4,
%captionpos=b,
basicstyle=\ttfamily\small,
keywordstyle=\color{orange},
commentstyle=\color{gray},
stringstyle=\color{dkgreen},
numberstyle=\tiny,
numbersep=8pt,
frame=single,%topline.bottomline,lines,single,leftline
identifierstyle=\color{dkblue},
numbers=left,
stepnumber=1,
xleftmargin=2em,xrightmargin=2em, aboveskip=1em,
breaklines=true
}
\usepackage[xetex,colorlinks]{hyperref}%394页  \href{网址}{文本}
\hypersetup{urlcolor=blue}
%\linebreak[2]%换行,152页
\usepackage{fancybox}%盒子宏包55页
\setcounter{secnumdepth}{4}
\CTEXoptions[contentsname={目\hspace{15pt}录}]
\CTEXsetup[beforeskip={-40pt},afterskip={20pt plus 2pt minus 2pt}]{chapter}

%目录设置
\usepackage{titletoc}
\usepackage[toc]{multitoc}
\titlecontents{chapter}[4em]{\addvspace{2.3mm}\bf}{\contentslabel{4.0em}}{}{\titlerule*[5pt]{$\cdot$}\contentspage}
\titlecontents{section}[4em]{}{\contentslabel{2.5em}}{}{\titlerule*[5pt]{$\cdot$}\contentspage}
\titlecontents{subsection}[7.2em]{}{\contentslabel{3.3em}}{}{\titlerule*[5pt]{$\cdot$}\contentspage}
\usepackage{fancyvrb}%75页抄录宏包
\begin{document}
\flushbottom%版心底部对齐
\newcommand{\dm}[1]{\colorbox{wgray}{\lstinline`#1`}}
\newcommand{\myroman}[1]{\uppercase\expandafter{\romannumeral#1}}
\newcounter{num}[section] \renewcommand{\thenum}{\arabic{num}.} \newcommand{\num}{\refstepcounter{num}\text{\thenum}}

\newenvironment{tips}{\kaishu\zihao{-6}\color{blue}{\noindent\rule[-3pt]{\textwidth}{0.5pt}\par \em \noindent {\zihao{-5} \textcolor[rgb]{1.00,0.00,0.00}{Tips}}}\par}{\\ \rule[3mm]{\textwidth}{0.5pt}\par}

\newenvironment{zhengming}{\kaishu\zihao{-5}\color{blue}{\noindent\em 证明:}\par}{\hfill $\diamondsuit$\par}

\tableofcontents
\pagenumbering{Roman}%设置目录页码
\clearpage
\pagenumbering{arabic}%设置正文页码

\chapter{泛型}
\section{泛型定义}

\section{泛型类}

泛型类的语法:
\begin{lstlisting}
class 类名称<泛型标识,泛型标识……>{
    private 泛型标识 变量名;
    ……
}
\end{lstlisting}
常用泛型标识:T、E、K、V

泛型类不支持基本数据类型,只支持类类型。

统一泛型类,根据不同的数据类型创建的对象,本质上是同一类型。(在逻辑上可以看成是多个不同的类型,但实际上都是相同的类型)
\section{从泛型类派生子类}
子类是泛型类,子类的泛型标识要和父类一致。(子类可以写多个泛型标识,即泛型扩展,只要父类的泛型类和子类中的一个一致即可)

泛型类派生子类的时候,子类不是泛型类,那么父类要明确数据类型。

\section{泛型接口}
\subsection{泛型接口的定义}
\begin{lstlisting}
interface 接口名称<泛型标识,泛型标识,……>{
    泛型标识 方法名();
}
\end{lstlisting}

\subsection{泛型接口的使用}
1.实现类不是泛型类,接口要明确数据类型

2.实现类也是泛型类,实现类和接口的泛型类要一致
\section{泛型方法}
在调用方法的时候指明泛型的具体类型。
\subsection{定义}
\begin{lstlisting}
修饰符 <T,E,...> 返回值类型 方法名(形参列表){
    方法体
}
\end{lstlisting}
\begin{itemize}
  \item public与返回值类型中间的<T>非常重要,可以理解为声明此方法为泛型方法
  \item 只有声明了<T>的方法才是泛型方法,泛型类中的使用了泛型的成员方法并不是泛型方法。
  \item <T>表明该方法将使用泛型类型T,此时才可以在方法中使用泛型类型T.
  \item 与泛型类的定义一样,此处T可以随便写为任意标识
\end{itemize}

可变参数的泛型方法。


泛型数组
\begin{lstlisting}
public class Fruit<>{
    private T[] array;
    
    public Fruit(Class<T> clz,int length){
        array = (T[])Array.newInstance(clz , length);
    }
}
\end{lstlisting}
\end{document} 