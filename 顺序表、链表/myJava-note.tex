\documentclass[a4paper]{report}
\usepackage[space,fancyhdr,fntef]{ctexcap}
\usepackage{fontspec}
\fontspec{宋体}
\setmainfont{Times New Roman}
%\fontsize{50pt}{50pt}\selectfont
\renewcommand{\rmdefault}{ptm}
\usepackage[namelimits,sumlimits,nointlimits]{amsmath}
\usepackage[text={169mm,250mm},bottom=20mm,top=25mm,left=25mm,right=15mm,centering]{geometry}
\usepackage{color}
\usepackage{CJKfntef}%下划线宏包160页
\usepackage{xcolor}
\usepackage{arydshln}%234页,虚线表格宏包
\pagestyle{fancy} \fancyhf{}
\fancyhead[OC]{\color{gray}\rightmark}

\fancyhead[EC]{\color{gray}\leftmark}
\fancyfoot[C]{\color{gray}\thepage}
\renewcommand{\headrule}{\color{gray}\hrule width\headwidth}
%\renewcommand{\footrulewidth}{0.4pt}%改为0pt即可去掉页脚上面的横线
%\usepackage{parskip}
%\usepackage{indentfirst}
\usepackage{graphicx}%插图宏包,参见手册318页
\definecolor{dkgreen}{RGB}{106,135,89}
\definecolor{dkblue}{RGB}{103,150,163}
\definecolor{wgray}{RGB}{248,248,248}
\definecolor{WGRAY}{RGB}{248,248,248}
\usepackage{listings}
\lstset{language=Java,
backgroundcolor=\color{wgray},
rulesepcolor=\color{red!20!green!20!blue!20},%代码块边框为淡青色
%lablestep=1,
%lablesep=5pt,
%lablestyle=\tiny,
%tablesize=4,
%captionpos=b,
basicstyle=\ttfamily\small,
keywordstyle=\color{orange},
commentstyle=\color{gray},
stringstyle=\color{dkgreen},
numberstyle=\tiny,
numbersep=8pt,
frame=single,%topline.bottomline,lines,single,leftline
identifierstyle=\color{dkblue},
numbers=left,
stepnumber=1,
xleftmargin=2em,xrightmargin=2em, aboveskip=1em,
breaklines=true
}
\usepackage[xetex,colorlinks]{hyperref}%394页  \href{网址}{文本}
\hypersetup{urlcolor=blue}
%\linebreak[2]%换行,152页
\usepackage{fancybox}%盒子宏包55页
\setcounter{secnumdepth}{4}
\CTEXoptions[contentsname={目\hspace{15pt}录}]
\CTEXsetup[beforeskip={-40pt},afterskip={20pt plus 2pt minus 2pt}]{chapter}

%目录设置
\usepackage{titletoc}
\usepackage[toc]{multitoc}
\titlecontents{chapter}[4em]{\addvspace{2.3mm}\bf}{\contentslabel{4.0em}}{}{\titlerule*[5pt]{$\cdot$}\contentspage}
\titlecontents{section}[4em]{}{\contentslabel{2.5em}}{}{\titlerule*[5pt]{$\cdot$}\contentspage}
\titlecontents{subsection}[7.2em]{}{\contentslabel{3.3em}}{}{\titlerule*[5pt]{$\cdot$}\contentspage}
\usepackage{fancyvrb}%75页抄录宏包
\begin{document}
\flushbottom%版心底部对齐
\newcommand{\dm}[1]{\colorbox{wgray}{\lstinline`#1`}}
\newcommand{\myroman}[1]{\uppercase\expandafter{\romannumeral#1}}
\newcounter{num}[section] \renewcommand{\thenum}{\arabic{num}.} \newcommand{\num}{\refstepcounter{num}\text{\thenum}}

\newenvironment{tips}{\kaishu\zihao{-6}\color{blue}{\noindent\rule[-3pt]{\textwidth}{0.5pt}\par \em \noindent {\zihao{-5} \textcolor[rgb]{1.00,0.00,0.00}{Tips}}}\par}{\\ \rule[3mm]{\textwidth}{0.5pt}\par}

\newenvironment{zhengming}{\kaishu\zihao{-5}\color{blue}{\noindent\em 证明:}\par}{\hfill $\diamondsuit$\par}

\tableofcontents
\pagenumbering{Roman}%设置目录页码
\clearpage
\pagenumbering{arabic}%设置正文页码
\chapter{顺序表}
\section{创建顺序表}
\begin{lstlisting}
//1.创建ArrayList实例
ArrayList<String> arrayList = new ArrayList<>();
//向上转型
List<String> arrayList = new ArrayList<>();
\end{lstlisting}
\section{往顺序表中添加元素}
\begin{lstlisting}[title=add一个参数]
//2.添加元素
//add一个参数版本的方法是把元素添加到顺序表的末尾
arrayList.add("C");
arrayList.add("C++");
arrayList.add("Java");
arrayList.add("Python");
\end{lstlisting}

\begin{lstlisting}[title=add两个参数]
//2.添加元素
//add两个参数版本的方法是把元素添加到顺序表的指定位置上
arrayList.add(2,"JavaScript");
//直接打印对象,会触发对象的toString方法
System.out.println(arrayList);
\end{lstlisting}
\section{删除元素}
\begin{lstlisting}
//3.删除元素
//按位置删除
arrayList.remove(2);
//按照值删除,如果有多个相同的值,只会删除第一个
arrayList.remove("JavaScript");
//特殊情况,删除整型的
List<Integer> arrayList2 = new ArrayList<>();
for (int i = 0; i < 5; i++) {
    arrayList2.add(i + 1);
}
arrayList2.remove(2);//删除的是下标为2的元素而不是值为2的
arrayList2.remove(Integer.valueOf(2));//这个才是删除值为2的元素
\end{lstlisting}
\section{查找}
\begin{lstlisting}
//4.查找

//查找是否存在
boolean ret  = arrayList.contains("Java");
System.out.println("查找Java的结果:"+ ret);
//查找具体的位置
int index = arrayList.indexOf("Java");
System.out.println("查找Java的位置:"+ index);
\end{lstlisting}
\section{获取/修改元素}
\begin{lstlisting}
//5.获取元素/修改元素
//获取元素
arrayList.get(0);//获取下标为0的元素
//修改元素
arrayList.set(0,"C#");//把下标为0的元素修改为C#
\end{lstlisting}
\section{迭代器:Iterator}
迭代器是用来遍历集合的一种重要手段,在非线性结构迭代器会大显身手
\begin{lstlisting}
//6.遍历操作
//通过下标遍历
for (int i = 0; i < arrayList.size(); i++) {
    System.out.println(arrayList.get(i));
}

//通过迭代器来遍历
//先通过iterator方法获取到迭代器对象
Iterator<String> iterator = arrayList.iterator();
//再通过while循环来进行遍历
while (iterator.hasNext()){
    String elem = iterator.next();
    System.out.println(elem);
}

//使用for-each来遍历
for (String str:arrayList) {
    System.out.println(str);
}
\end{lstlisting}
\section{练习:扑克牌游戏}
\begin{lstlisting}
//一张扑克牌
public class Card {
    //花色
    protected String suit;
    //点数
    protected String rank;

    public Card(String suit, String rank) {
        this.suit = suit;
        this.rank = rank;
    }
    @Override
    public String toString(){
        return "(" + this.suit + this.rank + ")";
    }
}


import java.util.ArrayList;
import java.util.Collections;
import java.util.List;
import java.util.Random;

public class PokerName {
    //存放4种花色
    public static final String[] suits = {"♥","♠","♣","♦"};
    //存放点数
    public static final String[] ranks = {"A","2","3","4","5","6","7","8","9","10","J","Q","K"};

    private static List<Card> buyPoker(){
        List<Card> poker = new ArrayList<>();
        for (int i = 0; i < suits.length; i++) {
            for (int j = 0; j < ranks.length ; j++) {
                poker.add(new Card(suits[i],ranks[j]));
            }
        }
        poker.add(new Card("red","Joker"));
        poker.add(new Card("black","Joker"));
        return poker;
    }

    private static void shuffle(List<Card> poker){
        Random random = new Random();
        for (int i = poker.size()-1; i > 0 ; i--) {
            int pos  = random.nextInt(i);
            swap(poker,i,pos);
        }
    }

    private static void swap(List<Card> poker, int i, int pos) {
        Card tmp = poker.get(i);
        poker.set(i,poker.get(pos));
        poker.set(pos,tmp);
    }

    public static void main(String[] args) {
        //1.创建出一副扑克牌
        List<Card> poker = buyPoker();
        System.out.println(poker);
        System.out.println("-------------------------------------");
        //2.洗牌
        shuffle(poker);
        System.out.println(poker);
        System.out.println("-------------------------------------");
        shuffle(poker);
        System.out.println(poker);
        System.out.println("-------------------------------------");
        //标准库自带的洗牌方法
        Collections.shuffle(poker);
        System.out.println(poker);
        //3.发牌:每个玩家发5张牌,这五张牌用另一个ArrayList表示
        List<List<Card>> players = new ArrayList<>();
        Scanner scanner = new Scanner(System.in);
        System.out.println("请输入玩家的个数:");
        int num = scanner.nextInt();
        for (int i = 0; i < num; i++) {
            players.add(new ArrayList<Card>());
        }
        for (int i = 0; i < 5; i++) {
            for (int j = 0; j < num; j++) {
                Card topCard = poker.remove(0);
                List<Card> player = players.get(j);
                player.add(topCard);
            }
        }
        System.out.println(players);
        System.out.println("-------------------------------------");
        //4.展示手牌:
        for (int i = 0; i < players.size(); i++) {
            System.out.println("第" + (i + 1) + "位玩家的牌是:" + players.get(i));
        }
    }
}
\end{lstlisting}
\end{document} 