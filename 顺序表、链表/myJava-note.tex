\documentclass[a4paper]{report}
\usepackage[space,fancyhdr,fntef]{ctexcap}
\usepackage{fontspec}
\fontspec{宋体}
\setmainfont{Times New Roman}
%\fontsize{50pt}{50pt}\selectfont
\renewcommand{\rmdefault}{ptm}
\usepackage[namelimits,sumlimits,nointlimits]{amsmath}
\usepackage[text={169mm,250mm},bottom=20mm,top=25mm,left=25mm,right=15mm,centering]{geometry}
\usepackage{color}
\usepackage{CJKfntef}%下划线宏包160页
\usepackage{xcolor}
\usepackage{arydshln}%234页,虚线表格宏包
\pagestyle{fancy} \fancyhf{}
\fancyhead[OC]{\color{gray}\rightmark}

\fancyhead[EC]{\color{gray}\leftmark}
\fancyfoot[C]{\color{gray}\thepage}
\renewcommand{\headrule}{\color{gray}\hrule width\headwidth}
%\renewcommand{\footrulewidth}{0.4pt}%改为0pt即可去掉页脚上面的横线
%\usepackage{parskip}
%\usepackage{indentfirst}
\usepackage{graphicx}%插图宏包,参见手册318页
\definecolor{dkgreen}{RGB}{106,135,89}
\definecolor{dkblue}{RGB}{103,150,163}
\definecolor{wgray}{RGB}{248,248,248}
\definecolor{WGRAY}{RGB}{248,248,248}
\usepackage{listings}
\lstset{language=Java,
backgroundcolor=\color{wgray},
rulesepcolor=\color{red!20!green!20!blue!20},%代码块边框为淡青色
%lablestep=1,
%lablesep=5pt,
%lablestyle=\tiny,
%tablesize=4,
%captionpos=b,
basicstyle=\ttfamily\small,
keywordstyle=\color{orange},
commentstyle=\color{gray},
stringstyle=\color{dkgreen},
numberstyle=\tiny,
numbersep=8pt,
frame=single,%topline.bottomline,lines,single,leftline
identifierstyle=\color{dkblue},
numbers=left,
stepnumber=1,
xleftmargin=2em,xrightmargin=2em, aboveskip=1em,
breaklines=true
}
\usepackage[xetex,colorlinks]{hyperref}%394页  \href{网址}{文本}
\hypersetup{urlcolor=blue}
%\linebreak[2]%换行,152页
\usepackage{fancybox}%盒子宏包55页
\setcounter{secnumdepth}{4}
\CTEXoptions[contentsname={目\hspace{15pt}录}]
\CTEXsetup[beforeskip={-40pt},afterskip={20pt plus 2pt minus 2pt}]{chapter}

%目录设置
\usepackage{titletoc}
\usepackage[toc]{multitoc}
\titlecontents{chapter}[4em]{\addvspace{2.3mm}\bf}{\contentslabel{4.0em}}{}{\titlerule*[5pt]{$\cdot$}\contentspage}
\titlecontents{section}[4em]{}{\contentslabel{2.5em}}{}{\titlerule*[5pt]{$\cdot$}\contentspage}
\titlecontents{subsection}[7.2em]{}{\contentslabel{3.3em}}{}{\titlerule*[5pt]{$\cdot$}\contentspage}
\usepackage{fancyvrb}%75页抄录宏包
\begin{document}
\flushbottom%版心底部对齐
\newcommand{\dm}[1]{\colorbox{wgray}{\lstinline`#1`}}
\newcommand{\myroman}[1]{\uppercase\expandafter{\romannumeral#1}}
\newcounter{num}[section] \renewcommand{\thenum}{\arabic{num}.} \newcommand{\num}{\refstepcounter{num}\text{\thenum}}

\newenvironment{tips}{\kaishu\zihao{-6}\color{blue}{\noindent\rule[-3pt]{\textwidth}{0.5pt}\par \em \noindent {\zihao{-5} \textcolor[rgb]{1.00,0.00,0.00}{Tips}}}\par}{\\ \rule[3mm]{\textwidth}{0.5pt}\par}

\newenvironment{zhengming}{\kaishu\zihao{-5}\color{blue}{\noindent\em 证明:}\par}{\hfill $\diamondsuit$\par}

\tableofcontents
\pagenumbering{Roman}%设置目录页码
\clearpage
\pagenumbering{arabic}%设置正文页码
\chapter{顺序表}
\section{顺序表的使用}
\subsection{创建顺序表}
\begin{lstlisting}
//1.创建ArrayList实例
ArrayList<String> arrayList = new ArrayList<>();
//向上转型
List<String> arrayList = new ArrayList<>();
\end{lstlisting}
\subsection{往顺序表中添加元素}
\begin{lstlisting}[title=add一个参数]
//2.添加元素
//add一个参数版本的方法是把元素添加到顺序表的末尾
arrayList.add("C");
arrayList.add("C++");
arrayList.add("Java");
arrayList.add("Python");
\end{lstlisting}

\begin{lstlisting}[title=add两个参数]
//2.添加元素
//add两个参数版本的方法是把元素添加到顺序表的指定位置上
arrayList.add(2,"JavaScript");
//直接打印对象,会触发对象的toString方法
System.out.println(arrayList);
\end{lstlisting}
\subsection{删除元素}
\begin{lstlisting}
//3.删除元素
//按位置删除
arrayList.remove(2);
//按照值删除,如果有多个相同的值,只会删除第一个
arrayList.remove("JavaScript");
//特殊情况,删除整型的
List<Integer> arrayList2 = new ArrayList<>();
for (int i = 0; i < 5; i++) {
    arrayList2.add(i + 1);
}
arrayList2.remove(2);//删除的是下标为2的元素而不是值为2的
arrayList2.remove(Integer.valueOf(2));//这个才是删除值为2的元素
\end{lstlisting}
\subsection{查找}
\begin{lstlisting}
//4.查找

//查找是否存在
boolean ret  = arrayList.contains("Java");
System.out.println("查找Java的结果:"+ ret);
//查找具体的位置
int index = arrayList.indexOf("Java");
System.out.println("查找Java的位置:"+ index);
\end{lstlisting}
\subsection{获取/修改元素}
\begin{lstlisting}
//5.获取元素/修改元素
//获取元素
arrayList.get(0);//获取下标为0的元素
//修改元素
arrayList.set(0,"C#");//把下标为0的元素修改为C#
\end{lstlisting}
\subsection{迭代器:Iterator}
迭代器是用来遍历集合的一种重要手段,在非线性结构迭代器会大显身手
\begin{lstlisting}
//6.遍历操作
//通过下标遍历
for (int i = 0; i < arrayList.size(); i++) {
    System.out.println(arrayList.get(i));
}

//通过迭代器来遍历
//先通过iterator方法获取到迭代器对象
Iterator<String> iterator = arrayList.iterator();
//再通过while循环来进行遍历
while (iterator.hasNext()){
    String elem = iterator.next();
    System.out.println(elem);
}

//使用for-each来遍历
for (String str:arrayList) {
    System.out.println(str);
}
\end{lstlisting}
\subsection{练习:扑克牌游戏}
\begin{lstlisting}
//一张扑克牌
public class Card {
    //花色
    protected String suit;
    //点数
    protected String rank;

    public Card(String suit, String rank) {
        this.suit = suit;
        this.rank = rank;
    }
    @Override
    public String toString(){
        return "(" + this.suit + this.rank + ")";
    }
}


import java.util.ArrayList;
import java.util.Collections;
import java.util.List;
import java.util.Random;

public class PokerName {
    //存放4种花色
    public static final String[] suits = {"♥","♠","♣","♦"};
    //存放点数
    public static final String[] ranks = {"A","2","3","4","5","6","7","8","9","10","J","Q","K"};

    private static List<Card> buyPoker(){
        List<Card> poker = new ArrayList<>();
        for (int i = 0; i < suits.length; i++) {
            for (int j = 0; j < ranks.length ; j++) {
                poker.add(new Card(suits[i],ranks[j]));
            }
        }
        poker.add(new Card("red","Joker"));
        poker.add(new Card("black","Joker"));
        return poker;
    }

    private static void shuffle(List<Card> poker){
        Random random = new Random();
        for (int i = poker.size()-1; i > 0 ; i--) {
            int pos  = random.nextInt(i);
            swap(poker,i,pos);
        }
    }

    private static void swap(List<Card> poker, int i, int pos) {
        Card tmp = poker.get(i);
        poker.set(i,poker.get(pos));
        poker.set(pos,tmp);
    }

    public static void main(String[] args) {
        //1.创建出一副扑克牌
        List<Card> poker = buyPoker();
        System.out.println(poker);
        System.out.println("-------------------------------------");
        //2.洗牌
        shuffle(poker);
        System.out.println(poker);
        System.out.println("-------------------------------------");
        shuffle(poker);
        System.out.println(poker);
        System.out.println("-------------------------------------");
        //标准库自带的洗牌方法
        Collections.shuffle(poker);
        System.out.println(poker);
        //3.发牌:每个玩家发5张牌,这五张牌用另一个ArrayList表示
        List<List<Card>> players = new ArrayList<>();
        Scanner scanner = new Scanner(System.in);
        System.out.println("请输入玩家的个数:");
        int num = scanner.nextInt();
        for (int i = 0; i < num; i++) {
            players.add(new ArrayList<Card>());
        }
        for (int i = 0; i < 5; i++) {
            for (int j = 0; j < num; j++) {
                Card topCard = poker.remove(0);
                List<Card> player = players.get(j);
                player.add(topCard);
            }
        }
        System.out.println(players);
        System.out.println("-------------------------------------");
        //4.展示手牌:
        for (int i = 0; i < players.size(); i++) {
            System.out.println("第" + (i + 1) + "位玩家的牌是:" + players.get(i));
        }
    }
}
\end{lstlisting}
\section{线性表的实现}
\subsection{空间管理}
根据实际需求动态调整,并同时保证高效率。

发生上溢时,适当的扩大数组的容量,扩容策略:倍增式扩容。累计时间复杂度为$O(n)$,分摊复杂度为$O(1)$。

\begin{lstlisting}
public class MyArrayList<E> {
    //属性
    private E[] data = null;
    //最大容量
    private int capacity = 4;
    //有效元素的个数
    private int size = 0;

    //构造方法
    public MyArrayList(){
        this.data = (E[])new Object[capacity];
    }

    //方法:增删改查

    //实现扩容
    private void realloc(){
        capacity <<= 1;
        E[] newData = (E[])new Object[capacity];
        for (int i = 0; i < size; i++) {
            newData[i] = data[i];
        }
        data = newData;
        newData = null;
    }

    //尾插
    public void add(E elem){
        if (size >= capacity){
            //如果满了,就需要扩容
            realloc();
        }
        data[size++] = elem;
    }
    //指定位置插入元素
    public void add(int index, E elem)throws IndexOutOfBoundsException{
        if(index < 0 || index > size){//等于size相当于是尾插
            throw new IndexOutOfBoundsException("下标越界");
        }
        if (size >= capacity){
            //如果满了,就需要扩容
            realloc();
        }
        for (int i = size - 1; i >= index; i--) {
            data[i + 1] = data[i];
        }
        data[index] = elem;
        size++;
    }

    //按照下标位置删除元素
    public E remove(int index) throws IndexOutOfBoundsException{
        if (index < 0){
            throw new IndexOutOfBoundsException("输入的下标小于下界(下界为0),你传入的下标是" + index + "。");
        }
        if (index >= size){
            String str = "输入的下标超过上界(上界为" + (size - 1) + "),你传入的下标是" + index + "。";
            throw new IndexOutOfBoundsException(str);
        }
        E elem = data[index];
        for (int i = index; i < size - 1; i++) {
            data[i] = data[i+1];
        }
        size--;
        return elem;
    }

    //按照元素的值来删除元素
    public boolean remove(E e){
        int index = 0;
        for (index = 0; index < size; index++) {
            if (data[index].equals(e)){
                break;
            }
        }
        if (index >= size){
            return false;
        }
        for (int i = index; i < size - 1; i++) {
            data[i] = data[i + 1];
        }
        size--;
        return true;
    }

    //根据下标获取元素
    public E get(int index)throws IndexOutOfBoundsException {
        if (index < 0){
            throw new IndexOutOfBoundsException("输入的下标小于下界(下界为0),你传入的下标是" + index + "。");
        }
        if (index >= size){
            String str = "输入的下标超过上界(上界为" + (size - 1) + "),你传入的下标是" + index + "。";
            throw new IndexOutOfBoundsException(str);
        }
        return data[index];
    }

    //根据下标修改元素
    public void set(int index, E e)throws IndexOutOfBoundsException {
        if (index < 0){
            throw new IndexOutOfBoundsException("输入的下标小于下界(下界为0),你传入的下标是" + index + "。");
        }
        if (index >= size){
            String str = "输入的下标超过上界(上界为" + (size - 1) + "),你传入的下标是" + index + "。";
            throw new IndexOutOfBoundsException(str);
        }
        data[index] = e;
    }

    //判断元素是否存在
    public boolean contains(E e){
        int i = 0;
        for (; i < size; i++) {
            if (data[i].equals(e)){
                break;
            }
        }
        if (i >= size){
            return false;
        }
        return true;
    }

    //查找元素位置
    public int indexOf(E e){

        for (int i = 0; i < size; i++) {
            if (data[i].equals(e)){
                return i;
            }
        }

        return -1;

    }

    //从后往前查找元素的位置
    public int lastIndexOf(E e){
        for (int i = size - 1; i >=0 ; i--) {
            if (data[i].equals(e)){
                return i;
            }
        }
        return -1;
    }

    //清空顺序表
    public void clear(){
        size = 0;
    }

    //表中元素的个数
    public int size(){
        return size;
    }

    //空表判断
    public boolean isEmpty(){
        return size == 0;
    }

    @Override
    public String toString() {
        if (size == 0){
            return "[]";
        }
        String str = "[";
        for (int i = 0; i < size - 1; i++) {
            str += data[i];
            str += ",";
        }
        str += data[size - 1];
        str += "]";
        return str;
    }
}
\end{lstlisting}
\chapter{链表}
单向、不带傀儡节点、带环的链表面试常见。

双向、带傀儡节点、带环的链表实际开发中常见。

\begin{lstlisting}
public class Node<E> {
    public E val;
    public Node<E> next;

    public Node(E val){
        this.val = val;
    }
}
\end{lstlisting}

\section{链表的使用}

\begin{lstlisting}
import java.util.Scanner;

public class Test {
    public static Node createList(){
        Node<Integer> a = new Node<>(1);
        Node<Integer> b = new Node<>(2);
        Node<Integer> c = new Node<>(3);
        Node<Integer> d = new Node<>(4);
        a.next = b;
        b.next = c;
        c.next = d;
        d.next = null;
        return a;
    }
    public static void main(String[] args) {
        Node head = createList();

        //遍历链表,打印链表的每个元素
        System.out.println("遍历链表,打印链表的每个元素");
        for (Node tmp = head; tmp != null; tmp = tmp.next) {
            System.out.println(tmp.val);
        }
        System.out.println("-----------------------------------------");
        //找链表的最后一个节点
        System.out.println("找链表的最后一个节点");
        Node cur = head;
        while(cur != null && cur.next != null){
            cur = cur.next;
        }
        System.out.println(cur.val);
        System.out.println("-----------------------------------------");
        //遍历链表,找到链表的倒数第二个节点
        System.out.println("遍历链表,找到链表的倒数第二个节点");
        cur = head;
        while(cur != null && cur.next != null && cur.next.next != null){
            cur = cur.next;
        }
        System.out.println(cur.val);
        System.out.println("-----------------------------------------");
        //取链表的第n个节点
        System.out.println("取链表的第n个节点:");
        cur = head;
        int index = 3;
        for (int i = 1; i < index; i++) {
            try{
                cur = cur.next;
            }catch(NullPointerException e){
                System.out.println("节点超出上限,该链表共有" + i + "个节点!");
                e.printStackTrace();
            }
        }
        try{
            System.out.println(cur.val);
        }catch(NullPointerException e){
            System.out.println("节点超出上限!");
            e.printStackTrace();
        }
        System.out.println("-----------------------------------------");
        //获取链表的长度
        System.out.println("获取链表的长度:");
        cur = head;
        int count = 0;
        for (;cur != null;cur = cur.next) {
            count++;
        }
        System.out.println("该链表共有" + count + "个节点!");
        System.out.println("-----------------------------------------");
        //遍历链表,是否存在某个元素
        System.out.println("遍历链表,是否存在某个元素:");
        cur = head;
        Scanner scanner = new Scanner(System.in);
        System.out.println("请输入要查找的元素:");
        int findNum = scanner.nextInt();
        for (;cur != null; cur = cur.next) {
            if (cur.val.equals(findNum)){
                break;
            }
        }
        if (cur != null){
            System.out.println("找到了!");
        }else {
            System.out.println("没找到!");
        }
    }
}
\end{lstlisting}

\section{链表的插入删除}

\begin{lstlisting}[title = 链表类]
public class Node {
    int val;
    Node next;
    public Node(int val){
        this.val = val;
    }
}
\end{lstlisting}

\begin{lstlisting}[title = 链表的插入删除操作]
public class Test {
    public static Node creatList(){
        Node a = new Node(1);
        Node b = new Node(2);
        Node c = new Node(3);
        Node d = new Node(4);
        a.next = b;
        b.next = c;
        c.next = d;
        d.next = null;
        return a;
    }

    //按值删除
    public static Node removeValue(Node head, int toDelete){
        if (head == null){
            return head;
        }
        if (head.val == toDelete){
            return head.next;
        }

        Node prev = head;
        while (prev != null && prev.next != null && prev.next.val != toDelete){
            prev = prev.next;
        }
        if (!(prev == null || prev.next == null)) {
            Node deleteNode = prev.next;
            prev.next = deleteNode.next;
            deleteNode.next = null;//回收删除节点的下一个指针
            return head;
        }
        return head;
    }
    //按节点位置删除1
    public static Node removeNode(Node head,Node toDelete){
        if (head == null){
            return head;
        }
        if (head == toDelete){
            return head.next;
        }
        Node prev = head;
        while(prev != null &&  prev.next != toDelete){
            prev = prev.next;
        }
        if (!(prev == null)){
            prev.next = toDelete.next;
            toDelete.next = null;
            toDelete = null;
            return head;
        }
        return head;
    }
    //按节点位置删除2
    public static Node removeNode1(Node head,Node toDelete) {
        if (head == null){
            return head;
        }
        if (head == toDelete){
            return head.next;
        }
        //无法删除最后一个节点
        if (toDelete.next != null){
            toDelete.val = toDelete.next.val;
            toDelete.next = toDelete.next.next;
            return head;
        }
        return head;
    }
    //统计有效值节点的个数
    public static int size(Node head){
        int count = 0;
        for (Node cur = head; cur != null ; cur = cur.next) {
            count++;
        }
        return count;
    }
    //按下标删除
    public static Node removeIndex(Node head, int index){
        //找到待删除节点的前一个位置index - 1
        if (index < 0){
            System.out.println("下标越界,下界为0。您输入的下界是" + index + "。");
            return head;
        }
        if(index >= size(head)){
            System.out.println("下标越界,上界为" + size(head) + "。您输入的下界是" + index + "。");
            return head;
        }
        if (index == 0){
            return head.next;
        }
        Node prev = head;
        for (int i = 0; i < index - 1; i++) {
            prev = prev.next;
        }
        prev.next = prev.next.next;
        return head;
    }

    public static Node removeTail(Node head){
        if (head == null){
            return head;
        }
        Node pre = head;
        while( pre != null &&pre.next.next!= null && pre.next != null){
            pre = pre.next;
        }
        if (!(pre.next == null)){
            pre = pre.next;
            return head;
        }
        return head;
    }
    //把数组转为链表
    public static Node arrayToLinkedList(int[] array){
        Node head = new Node(0);
        Node tmp = head;
        for (int i = 0; i < array.length; i++) {
            Node cur = new Node(array[i]);
            tmp.next = cur;
            tmp = tmp.next;
        }
        return head.next;
    }

    public static void main(String[] args) {
        Scanner scanner = new Scanner(System.in);
        Node head = creatList();
        Node insert = new Node(-3);
        Node pre = null;
        Node next = null;
//        System.out.println("请输入要插入元素的位置(比如在2和3之间插入,则输入2)");
//        int index = scanner.nextInt();
//        for(pre = head;pre != null;pre = pre.next){
//            if (index == pre.val){
//                next = pre.next;
//                break;
//            }
//        }
//        if (pre != null){
//            insert.next = next;
//            pre.next = insert;
//            insert = null;
//        }else {
//            System.out.println("下标越界,无法插入!");
//        }
//        for (pre = head;pre != null;pre = pre.next){
//            System.out.println(pre.val);
//        }
        //节点插入到链表的头部
        insert.next = head;
        head = insert;
        for (pre = head;pre != null;pre = pre.next){
            System.out.println(pre.val);
        }
    }
}
\end{lstlisting}
\end{document} 